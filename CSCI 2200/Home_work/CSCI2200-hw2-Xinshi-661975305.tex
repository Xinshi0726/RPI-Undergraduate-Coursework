\documentclass{article}
\title{CSCI 2200 HW1}
\author{Xinshi,Wang}
\usepackage[letterpaper,textwidth=5.5in,right=0.6in,textheight=9in,left=0.6in,top=0.7in,bottom=0.7in]{geometry}
\usepackage{scrextend}
\usepackage{graphicx}
\usepackage{xcolor}
\usepackage{amssymb}
\usepackage{amsmath}
\begin{document}
\noindent
CSCI 2200 HW2\\
Wang Xinshi\\

%Q 3.6 starts from here
\begin{addmargin}[2em]{2em}
	Problem 3.6. Kilam’s has 2GB of RAM Liamsi has 4GB of RAM. Which propositions are true? \\
	\indent (a) if Kilam has more RAM than Liamsi then pigs fly. (b) if Liamsi has more RAM than Kilam then pigs fly. \\
	\indent (c) Kilam has more RAM than Liamsi and pigs fly. (d) Kilam has more RAM than Liamsi or pigs fly.\\
	\indent (e) Liamsi has more RAM than Kilam and pigs fly. (f) Liamsi has more RAM than Kilam or pigs fly.\\
	
	\textcolor{blue}{
		The case $p \implies q$ is false is when p is T and q is F. We know that pigs can fly is invalid, so p must be False. Therefore $(a)$ is correct.}
	
	\textcolor{blue}{
		The statement $p \land q$ is valid is only when p and q are both T. Since pigs cannot fly, no statment using and is correct.}
	
	\textcolor{blue}{
		The statement $p \lor q$ is valid if either p or q is correct. Since pigs cannot fly, q is false. Therefore p has to be true. Hence $(f)$ is correct.
	}

	\textcolor{blue} {
		Therefore, statement $(a)$ and statement $(f)$ is correct.
	}

\end{addmargin}
%Q 3.6 ends here 

\clearpage

%Q 3.46 starts from here
\begin{addmargin}[2em]{2em}
	Problem 3.46. Formulate the appropriate predicates, identify the domain of the predicate and give the “mathematical” version of the following statements.\\
	 \indent (a) Every person has at most one job. \\
	 \indent (b) Kilam has some grey hair.\\
	 \indent (c) Everyone has some grey hair. \\
	 \indent (d) Everyone is a friend of someone. \\
	 \indent (e) All professors consider their students as a friend. \\
	 \indent (f) No matter what integer you choose, there is always an integer that is larger.\\
	 
	 \textcolor{blue}{
	 	$(a)$
	 	\begin{addmargin}[4em]{0em}
	 		 $p=\{p \mid \text{p is a person}\}$\\
	 		 $p(A) = \text{person A has at most one job}$\\
	 		 $\forall x \in p : P(A)$
	 	\end{addmargin}
	 }
 	
 	 \textcolor{blue}{
 		$(b)$
 		\begin{addmargin}[4em]{0em}
 			$k=\{k \mid \text{k is a person}\}$\\
 			$p(A) = \text{person A has some grey hair}$\\
 			$Q(A) = \text{person A is kilam}$\\
 			$\exists x \in k : (Q(A) \land P(A))$
 		\end{addmargin}
 	}
 
 	 \textcolor{blue}{
 		$(c)$
 		\begin{addmargin}[4em]{0em}
 			$k=\{k \mid \text{k is a person}\}$\\
 			$p(A) = \text{person A has some grey hair}$\\
 			$\forall x \in k : P(A)$
 		\end{addmargin}
 	}
 
 	 \textcolor{blue}{
 		$(d)$
 		\begin{addmargin}[4em]{0em}
 			$k=\{k \mid \text{k is a person}\}$\\
 			$p(a,b) = \text{person a is a friend of person b}$\\
 			$\forall x \in k :(\exists y \in k:p(x,y))$
 		\end{addmargin}
 	}
 	
 	 \textcolor{blue}{
 		$(e)$
 		\begin{addmargin}[4em]{0em}
 			$k=\{k \mid \text{k is a professor}\}$\\
 			$s = \{s \mid \text{s is a student}\} $\\
 			$p(a,b) = \text{person a is a student of professor b}$\\
 			$q(a,b) = \text{person a considers person b as a friend}$\\
 			$\forall x \in k : (\forall y \in s : p(y,x) \implies q(x,y))$
 		\end{addmargin}
 	}
 
	 \textcolor{blue}{
	 	$(e)$
	 	\begin{addmargin}[4em]{0em}
	 		$I = \{i \mid \text{i is an integer} \}$\\
	 		$g(a,b) = \text{integer a is greater than integer b}$\\
	 		$\forall x \in I : (\exists y \in I:g(y,x))$
	 	\end{addmargin}
 }
\end{addmargin}

%Q 3.46 ends here

\clearpage

%Q 3.57 starts from here
\begin{addmargin}[2em]{2em}
	Problem 3.57. True or false. A counterexample to “all ravens are black” must be: (a) Non-raven. (b) Non-black.
	
	\textcolor{blue}{
		\begin{addmargin}[2em]{2em}
			The negation of a for all statement should be a there exists statement. Therefore the counter example should be there exists a raven that is not black. Therefore choose $(b)$.
		\end{addmargin}
	}
\end{addmargin}
%Q 3.57 ends here

\clearpage
%Q 4.16(d) starts from here
\begin{addmargin}[2em]{2em}
	Problem 4.16(d). Prove $\forall (a,b) \in \mathbb{R}^2:(a \neq b) \implies (a^2 + b^2 > 2ab ) $\\
	\textcolor{blue}{
		\begin{addmargin}[2em]{1em}
			A direct proof could prove this problem.\\
			we must show that when $a \neq b$ is True, $a^2+b^2 > 2ab$ could not be False.\\
			Assume $a \neq b$ is True, we then have $(a-b)^2 > 0$. Since $a$ and $b$ are not equal the square could not be $0$.\\
			$(a-b)^2$ is the same as $a^2 -2ab +b^2$ which is also greater than 0. This means $a^2 - 2ab +b^2 > 0$. Moving the 2ab to the right side gives us $a^2+b^2 > 2ab$.\\
			Therefore the statement $a^2 + b^2 > 2ab$ is True when $a \neq b$.
			\begin{flushright}
				$\square$
			\end{flushright}
		\end{addmargin}
}
\end{addmargin}
%Q 4.16(d) ends here

\clearpage

%Q 4.23 starts from here
\begin{addmargin}[2em]{2em}
Problem 4.23. A triangle is drawn on the plane. The vertices of the triangle have integer coordinates. Prove that the triangle cannot be equilateral.
	\textcolor{blue}{
	\begin{addmargin}[2em]{1em}
		A contradiction proof could prove this problem.\\
		Assume there's an equilateral triangle on the vertices of the integer coordinates, then we know that for the three sides of the triangle are all equal.\\
		Let point A be $(0,0)$, point B be $(a,b)$ and point C be $(c,d)$. We can know that $L = a^2+b^2 = c^2 + d^2 = (a-c)^2+(b-d)^2$. Then $2L = a^2 + b^2 + c^2 + d^2$. From the last term we know that $L = a^2 - 2ac +c^2 + b^2 -2bd + d^2$. substract $2L$ by $L$ gives $L = 2ac + 2bd$. Therefore $L = a^2+b^2 = c^2 + d^2 = 2ac+2bd$.\\
		From Brahmagupta–Fibonacci identity we know that $(a^2+b^2)(c^2+d^2) = (ac+bd)^2+(ad-bc)^2$. Let $L_1$ be $K$, therefore the identity could be expressed as $K^2=\frac{K^2}{4}+(ad-bc)^2$ which means $3K^2 = 4(ad-bc)^2$. 
		Therefore $K^2 = \frac{4}{3}(ad-bc)^2$, which means $K = \pm \sqrt{\frac{4}{3}}(ad-bc)$. Hence, we found that K is irational since $\sqrt{\frac{4}{3}}$ is irational. However, $K = 2ac + 2bd$ where $a,b,c,d$ are all positive integers which means $K$ must be rational. We found a contradiction.\\
		Therefore, there does not exist an equilateral triangle on the vertifes of the integer coordinates.
		\begin{flushright}
			$\square$
		\end{flushright}
	\end{addmargin}
}
\end{addmargin}
%Q 4,23 ends here
\end{document}