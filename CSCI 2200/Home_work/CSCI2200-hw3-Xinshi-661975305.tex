\documentclass{article}
\title{CSCI 2200 HW1}
\author{Xinshi,Wang}
\usepackage[letterpaper,textwidth=5.5in,right=0.6in,textheight=9in,left=0.6in,top=0.7in,bottom=0.7in]{geometry}
\usepackage{scrextend}
\usepackage{graphicx}
\usepackage{xcolor}
\usepackage{amssymb}
\usepackage{amsmath}
\usepackage{setspace}
\begin{document}
\noindent
CSCI 2200 HW3\\
Wang Xinshi\\

%Q 4.27(c)(d) starts from here
\begin{addmargin}[2em]{2em}
	Problem 4.27(c) \quad $\forall n:(P(n) \implies Q(n)$
	
	\textcolor{blue}{
		Prove: You can prove it use direct proof. Show that Q(n) cannot be false when P(n) is true.}
	
	\textcolor{blue}{
		Disprove: You can disprove it by showing that Q(n) is false when P(n) is true. 
		}
	
	\noindent Problem 4.27(d) \quad $\forall x:((\forall n:(P(n))) \implies Q(x))$
	
	\textcolor{blue}{
		Prove:Firstly prove $P(n)$ is correct because for all $n$, $P(n)$ must be true. Since the first part of the if then statement is correct, $Q(x)$ must also be true}
	
	\textcolor{blue}{
		Disprove: Show either $P(n)$ is false, or $Q(n)$ is false.
	}

\end{addmargin}
%Q 4.27(c)(d) ends here 

\clearpage

%Q 5.12(h) starts from here
\begin{addmargin}[2em]{2em}
	Problem 5.12(h) For $n \geq -1$, prove by induction:
	\begin{center}
		$10^0 + 10^1 + 10^2 + ... +10^n < 10^{n+1}$
	\end{center}
	
	\textcolor{blue}{
		Base Case: n = 1
		\begin{addmargin}[2.3cm]{0em}
				$10^0 + 10^1 = 11$\\
			    $10^2 = 100$\\
			    $\because 11 \le 100$\\
			    $\therefore \text{the base case holds.}$	
		\end{addmargin}}
	
	\textcolor{blue}{
		Inductive step:
		\begin{addmargin}[3cm]{0em}
			Assume: $10^0+10^1+...+10^n \le 10^{n+1}$ holds\\
			Prove: $10^0 + 10^1 +....+ 10^{n+1} \le 10^{n+2}$
		\end{addmargin}}
	
	\textcolor{blue}{
		\begin{addmargin}[3.5cm]{0em}
			$10^0 + 10^1 +... +10^n+1 \le 10^{n+1} + 10^{n+1}$\\
			$10^0 + 10^1 +... +10^n+1 \le 2 \cdot 10^{n+1}$\\
			$10^{n+2} = 10^{n+1} \cdot 10 = 10 \cdot 10^{10^n+1}$\\
			$10^{n+2} = 10 ^{n+1} \cdot 10 = 10 \cdot 10^{n+1}$\\
			$\because 10 \cdot 10^{n+1} \ge 2 \cdot 10^{n+1}$
			$\therefore 10^0 + 10^1 +... +10^n+1 \le 10 \cdot 10^{n+1}\\ \le 10^{n+2}$\\
			$\therefore 10^0 + 10^1 +... +10^n+1 \le 10^{n+2}$\\
		\end{addmargin}
		\begin{flushright}
			$\square$
		\end{flushright}
	}
Problem 5.12(J) is on the next page
\end{addmargin}

%Q 5.12(h) ends here

\clearpage

%Q 5.12(J) starts from here
\begin{addmargin}[2em]{2em}
	Problem 5.12(j) For $n \geq -1$, prove by induction:
	\begin{center}
		$(1+x)^n \geq 1+nx$ when $x \geq -1$
	\end{center} 
	
	\textcolor{blue}{
	Base Case: n = 1
	\begin{addmargin}[2.3cm]{0em}
		1+x = 1+x
		$\therefore \text{the base case holds.}$	
\end{addmargin}}

	\textcolor{blue}{
		Inductive step:
		\begin{addmargin}[2.35cm]{0em}
			Assume: $(1+x)^n \geq 1+nx$ holds\\
			Prove: $(1+x)^{n+1} \geq 1+(n+1)x$
	\end{addmargin}}
	
	\textcolor{blue}{
		\begin{addmargin}[3.5cm]{0em}
			$(1+x)^{n} \cdot (1+x) \geq (1+nx) \cdot (1+x)$\\
			$(1+x)^{n+1} \geq (1+x+nx+nx^2)$\\
			$\because (1+x+nx+nx^2) - (1+nx+x) = nx^2$ and $nx^2$ because $x^2 \geq 0$ and $n \geq 1$\\
			$\therefore 1+x+nx+nx^2 \geq (1+nx+x)$\\
			$\therefore (1+n)^{n+1} \geq 1+nx+x$
		\end{addmargin}
		\begin{flushright}
			$\square$
		\end{flushright}
	}
\end{addmargin}
%Q 5.12(J) ends here

\clearpage
%Q 5.26(f) starts from here
\begin{addmargin}[2em]{2em}
	Problem 5.26(f). Find a formula for the quantity of interest:
	\begin{center}
		$$S(n) = \sum_{i=1}^{n} \dfrac{n}{(n+1)!}$$
	\end{center}
	\textcolor{blue}{
		\begin{addmargin}[2em]{2em}
			\begin{spacing}{2.2}
				$S(1) = \dfrac{1}{2!} = \dfrac{1}{2}$\\
				$S(2) = \dfrac{1}{2!}+\dfrac{2}{3!} = \dfrac{1}{2} + \dfrac{2}{6} = \dfrac{5}{6}$\\
				$S(3) = \dfrac{1}{2!} + \dfrac{2}{3!} + \dfrac{3}{4!} = \dfrac{5}{6} + \dfrac{3}{24} = \dfrac{20}{24} + \dfrac{3}{24} = \dfrac{23}{24}$
			\end{spacing}
		\noindent The Assumption here is $$\sum_{i=1}^{n} \dfrac{n}{(n+1)!} = \dfrac{(n+1)!-1}{(n+1)!}$$ \\
		Base case: $$S(1) = \dfrac{(1+1)!-1}{(1+1)!} = \dfrac{1}{2}$$
		\begin{center}
			We have verified the base case is valid
		\end{center}
		Assume: $$\sum_{i=1}^{n} \dfrac{n}{(n+1)!} = \dfrac{(n+1)!-1}{(n+1)!}$$
		Prove: $$\sum_{i=1}^{N+1} \dfrac{n+1}{(n+2)!} = \dfrac{(n+2)!-1}{(n+2)!}$$
		$$\dfrac{1}{2!}+\dfrac{2}{3!}+\dfrac{3}{4!}+...+\dfrac{n+1}{(n+2)!} = \dfrac{(n+1)!-1}{(n+1)!}+\dfrac{n+1}{(n+2)!}$$ 
		$$\dfrac{1}{2!}+\dfrac{2}{3!}+\dfrac{3}{4!}+...+\dfrac{n+1}{(n+2)!} = \dfrac{((n+1)!-1) \cdot (n+2)}{(n+2)!}+\dfrac{n+1}{(n+2)!}$$ 
		$$\dfrac{1}{2!}+\dfrac{2}{3!}+\dfrac{3}{4!}+...+\dfrac{n+1}{(n+2)!} = \dfrac{((n+2)!-(n+2))+(n+1)}{(n+2)!}$$ 
		$$\dfrac{1}{2!}+\dfrac{2}{3!}+\dfrac{3}{4!}+...+\dfrac{n+1}{(n+2)!} = \dfrac{((n+2)!-1}{(n+2)!}$$ 
		$$\therefore \sum_{i=1}^{N+1} \dfrac{n+1}{(n+2)!} = \dfrac{(n+2)!-1}{(n+2)!}$$
		\begin{flushright}
			$\square$
		\end{flushright}
		\end{addmargin}	}
\end{addmargin}
%Q 5.26(f) ends here

\clearpage

%Q 5.71 starts from here
\begin{addmargin}[2em]{2em}
Problem 5.71.  Distinct numbers $X_1,...,X_n$ must be placed in n boxes separated by $<$ and $>$ signs so that all inequalities are obeyed between consecutive numbers. For example, $5,7,3,8$ can be placed in $\square$ $<$ $\square$ $>$ $\square$ $<$ $\square$ as follows, $5 < 8 > 3 < 7$ . Prove this can always be done no matter what the numbers and inequalities are.

	\textcolor{blue}{
	\begin{addmargin}[2em]{1em}
		$P(n):$ $n$ distinct numbers can be placed in boxes separeted by $n-1$ inequality signs for all $n \in \mathbb{N}.$\\\\
		Base case: $P(1):$ $1$ distinct number can be placed in boxes separeted by $0$ inequality sign.\\
		\begin{addmargin}[1.7cm]{0cm}
			Let us pick an arbitray number $X$. We can prove that X can be placed on its own with 1 box and no inequality signs.
			$$X$$
			Therefore we have verified the base case.\\
		\end{addmargin}
		Inductive step: 
		\begin{addmargin}[2.35cm]{0cm}
			Assume: $P(n)$ holds\\
			Prove: $P(n+1):$ $n+1$ distinct numbers can be placed in boxes separeted by $n$ inequality signs for all $n+1 \in \mathbb{N}.$\\
			\begin{addmargin}[0cm]{0cm}
				The inequality connecting $Box_{n-1}$ and $box_{n}$ can either be $>$ or $<$. Consider two exhuastive cases.
				\begin{addmargin}[1.5cm]{0cm}
					Case 1: The inequality sign is $<$. Find the largest number $X_{max}$ in $X_1,X_2,...X_n$. Then, the other $n$ numbers can be placed in the first n boxes(assumption in inductive steps). Since $X_{max}$ is larger than any number in $X_1,X_2,...X_n$, we can always place it in the last box since the last inequality sign is $<$.\\\\
					Case 2: The inequality sign is $>$. Find the largest number $X_{min}$ in $X_1,X_2,...X_n$. Then, the other $n$ numbers can be placed in the first n boxes(assumption in inductive steps). Since $X_{min}$ is smaller than any number in $X_1,X_2,...X_n$, we can always place it in the last box since the last inequality sign is $>$.\\
				\end{addmargin}
				Hence, we have shown that p(n+1) is true when p(n) is true.
			\end{addmargin}
		\end{addmargin}
		\begin{flushright}
			$\square$
		\end{flushright}
	\end{addmargin}
}
\end{addmargin}
%Q 5.71 ends here

\clearpage

\begin{addmargin}[2em]{2em}
Problem 6.6. Let $H_n = 1/1 + 1/2 +···+ 1/n$, the nth Harmonic number, and $S_n = H_1/1 + H_2/2 +···+ H_n/n$. (a) Prove $S_n \leq H_n^2/2 + 1$ by induction. What goes wrong? (b) Prove the stronger claim $S_n \leq H_n^2/2 + (1/12 + 1/22 +···+ 1/n^2)/2$. Why is this stronger? 
	\textcolor{blue}{
		\begin{addmargin}[2em]{1em}
			(a). Base case: $n= 1$, $H_1 = 1$ , $S_1 = H_1/1 =1$, $\therefore S_1 \leq H_n^2/2+1$\\
			Inductive Step:
			\begin{addmargin}[2.35cm]{0cm}
				Assume: $S_n \leq H_n^2/2 + 1$\\
				Prove: $S_{n+1} \leq H_{n+1}^2/2 + 1$
				$$H_{n+1}^2=\dfrac{H_{n}^2}{2}+\dfrac{H_n}{n+1}+\dfrac{1/2}{(n+1)^2}$$
				$$S_{n+1} \leq \dfrac{H_n^1}{2}+1+\dfrac{H_{n+1}}{n+1}$$
				$$S_{n+1} \leq \dfrac{H_n^1}{2}+1+\dfrac{H_{n}+\dfrac{1}{n+1}}{n+1}+1$$
				$$S_{n+1} \leq \dfrac{H_n^1}{2}+1+\dfrac{H_{n}+\dfrac{1}{n+1}}{n+1}+1$$
				$$S_{n+1} \leq \dfrac{H_n^2}{2}+1+\dfrac{H_{n}}{n+1}+\dfrac{1}{(n+1)^2}+1$$
				$$\because \dfrac{H_n^2}{2}+1+\dfrac{H_{n}}{n+1}+\dfrac{1}{(n+1)^2} \geq  \dfrac{H_{n}^2}{2}+\dfrac{H_n}{n+1}+\dfrac{1/2}{(n+1)^2}$$
				$$\therefore S_{n+1} \nleq H_{n+1}^2/2 + 1$$
			\end{addmargin}
		\begin{spacing}{2.2}
				(b). Base case: $n= 1$, $H_1 = 1$ , $S_1 = H_1/1 =1$, $\therefore S_1 \leq H_n^2/2+\dfrac{\dfrac{1}{1^2}}{2}$\\
				Inductive Step:
				\begin{addmargin}[2cm]{0cm}
					Assume: $S_n \leq \dfrac{H_n^2}{2}+\dfrac{(\dfrac{1}{1^2}+\dfrac{1}{2^2}+...+\dfrac{1}{n^2})}{2}$\\
					Prove: $S_{n+1} \leq \dfrac{H_{n+1}^2}{2}+\dfrac{(\dfrac{1}{1^2}+\dfrac{1}{2^2}+...+\dfrac{1}{(n+1)^2})}{2}$
					$$S_{n+1} \leq \dfrac{H_n^2}{2}+\dfrac{(\dfrac{1}{1^2}+\dfrac{1}{2^2}+...+\dfrac{1}{n^2})}{2}+\dfrac{H_{n+1}}{n+1}$$
					$$S_{n+1} \leq \dfrac{H_n^2}{2}+\dfrac{(\dfrac{1}{1^2}+\dfrac{1}{2^2}+...+\dfrac{1}{n^2})}{2}+\dfrac{H_{n}+\dfrac{1}{n+1}}{n+1}$$
					$$S_{n+1} \leq \dfrac{H_n^2}{2}+\dfrac{(\dfrac{1}{1^2}+\dfrac{1}{2^2}+...+\dfrac{1}{n^2})}{2}+\dfrac{H_{n}}{n+1}+\dfrac{1}{(n+1)^2}$$
					$$S_{n+1} \leq \dfrac{H_n^2}{2}+\dfrac{(\dfrac{1}{1^2}+\dfrac{1}{2^2}+...+\dfrac{1}{n^2})}{2}+\dfrac{H_{n}}{n+1}+\dfrac{1/2}{(n+1)^2}++\dfrac{1/2}{(n+1)^2}$$
					$$S_{n+1} \leq \dfrac{H_{n+1}^2}{2}+\dfrac{(\dfrac{1}{1^2}+\dfrac{1}{2^2}+...+\dfrac{1}{(n+1)^2})}{2}$$\\
					It is stronger because it is more precise. It replace a sequence of number greater than one.
						\begin{flushright}
						$\square$
					\end{flushright}
				\end{addmargin}
		\end{spacing}
		\end{addmargin}
	}
\end{addmargin}
\end{document}