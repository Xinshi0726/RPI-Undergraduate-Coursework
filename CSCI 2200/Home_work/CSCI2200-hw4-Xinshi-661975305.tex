\documentclass{article}
\title{CSCI 2200 HW1}
\author{Xinshi,Wang}
\usepackage[letterpaper,textwidth=5.5in,right=0.6in,textheight=9in,left=0.6in,top=0.7in,bottom=0.7in]{geometry}
\usepackage{scrextend}
\usepackage{graphicx}
\usepackage{xcolor}
\usepackage{amssymb}
\usepackage{amsmath}
\usepackage{setspace}
\begin{document}
\noindent
CSCI 2200 HW4\\
Wang Xinshi\\

%Q 7.45 starts from here
\begin{addmargin}[2em]{2em}
	Problem 7.45: Give recursive definitions for the set S in each of the following cases.\\
	(a) $S = \{0, 3, 6, 9, 12, . . .\}$, the multiples of 3.\\
	(b) $S = \{1, 2, 3, 4, 6, 7, 8, 9, 11, . . .\}$, the numbers which are not multiples of 5.\\
	(c) $S$ = $\{$all strings with the same number of 0’s as 1’s$\}$ (e.g. $0011,0101,100101$).\\
	(d) The set of odd multiples of $3$.\\
	(e) The set of binary strings with an even number of $0$’s.\\
	(f) The set of binary strings of even length.
\\
	
	%(a) starts from here
	\noindent
	\textcolor{blue}{
	(a).$S = \{\text{The multiples of 3}\}$
	}	
	
	\textcolor{blue}{
		$\textcircled{1}. 0 \in S$ 
	}

	\textcolor{blue}{
		$\textcircled{2}. x \in S \implies x+3 \in S$ 
	}
	
	\textcolor{blue}{
		$\textcircled{3}.$ Nothing else is in $S$.\\
	}
	%(a) ends here
	
	%(b) starts from here
	\noindent
	\textcolor{blue}{
	(b).$S = \{\text{The numbers which are not multiples of 5}\}$
	}

	\textcolor{blue}{
	$\textcircled{1}. 1 \in S$
	}
	
	\textcolor{blue}{
	$\textcircled{2}. x \in S \implies 2x,3x,4x  \in S$
	}

	\textcolor{blue}{
	$\textcircled{4}.$ Nothing else is in $S$.\\
	}
	%(b) ends here 
	
	%(c) starts from here
	\noindent
	\textcolor{blue}{
		(c).$S = \{\text{All strings with the same number of 0's as 1's}\}$
	}
	
	\textcolor{blue}{
		$\textcircled{1}. \epsilon \in S$
	}
	
	\textcolor{blue}{
		$\textcircled{2}. x \in S \implies 01x \in S$
	}
	
	\textcolor{blue}{
		$\textcircled{3}. x \in S \implies x01 \in S$
	}

	\textcolor{blue}{
		$\textcircled{4}. x \in S \implies 10x \in S$
	}
	
	\textcolor{blue}{
		$\textcircled{5}. x \in S \implies x10 \in S$
	}
	
	\textcolor{blue}{
		$\textcircled{6}. x \in S \implies 0x1 \in S$
	}

	\textcolor{blue}{
		$\textcircled{7}. x \in S \implies 1x0 \in S$
	}

	\textcolor{blue}{
		$\textcircled{8}.$ Nothing else is in $S$.\\
	}
	%(c) ends here 
	
	%(d) starts from here
	\noindent
	\textcolor{blue}{
		(d).$S = \{\text{The set of odd multiples of 3}\}$
	}
	
	\textcolor{blue}{
		$\textcircled{1}. 3 \in S$
	}
	
	\textcolor{blue}{
		$\textcircled{2}. x \in S \implies (2k+1)x  \in S$, where $k \in \mathbb{N}$
	}
	
	\textcolor{blue}{
		$\textcircled{3}.$ Nothing else is in $S$.\\
	}
	%(d) ends here 
	
	%(e) starts from here
	\noindent
	\textcolor{blue}{
		(e).$S = \{\text{All strings with an even number of 0's}\}$
	}
	
	\textcolor{blue}{
		$\textcircled{1}. \epsilon,1 \in S$
	}
	
	\textcolor{blue}{
		$\textcircled{2}. x \in S \implies 0x0 \in S$
	}
	
	\textcolor{blue}{
		$\textcircled{2}. x,y \in S \implies xy \in S$
	}

	\textcolor{blue}{
		$\textcircled{8}.$ Nothing else is in $S$.\\
	}
	%(e) ends here 
	
	%(f) starts from here
	\noindent
	\textcolor{blue}{
		(f).$S = \{\text{The set of strings with even length}\}$
	}
	
	\textcolor{blue}{
		$\textcircled{1}. \epsilon \in S$
	}
	
	\textcolor{blue}{
		$\textcircled{2}. x \in S \implies x00 \in S$.
	}
	
	\textcolor{blue}{
		$\textcircled{2}. x \in S \implies x01 \in S$.
	}
	
	\textcolor{blue}{
		$\textcircled{2}. x \in S \implies x10 \in S$.
	}
	
	\textcolor{blue}{
		$\textcircled{2}. x \in S \implies x11 \in S$.
	}

	\textcolor{blue}{
		$\textcircled{3}.$ Nothing else is in $S$.\\
	}
	%(f) ends here
\end{addmargin}
%Q 7.45 ends here 

\newpage

%Q 7.60 starts from here
\begin{addmargin}[2em]{2em}
	Problem 8.19(c). prove that every RFBT has an odd number of vertices.
	
	\noindent
	\textcolor{blue}{
	$\textcircled{1}. \epsilon \in RFBT$\\
	$\textcircled{2}. T_1,T_2 \in RFBT \implies T_1, T2$ form a new RFBT.\\
	}
	
	\noindent
	\textcolor{blue}{
		Base Case: $\epsilon + F = 3 $ vertcies, which is odd.\\
		Inductive step: The property preserves in the child.\\
		(a). $T_1 = \epsilon, T_2 = \epsilon$.\\
		\quad $n = 0$.\\
		The property preserves.\\\\
		(b) $T_1 = \epsilon, T_2 = F$\\
			$n = 2 \times 1 +1 = 3$\\
			The property preserves. \\\\
		(b) $T_1 = F, T_2 = \epsilon$\\
		$n = 2 \times 1 +1 = 3$\\
		The property preserves. \\\\
		(b) $T_1 = F, T_2 = F$\\
		$n = 2 \times 2 +1 = 5$\\
		The property preserves. 
	}
\end{addmargin}

%Q 7.60 ends here

\clearpage

%Q 9.3(h) starts from here
\begin{addmargin}[2em]{2em}
	Problem 9.3(h)
	\begin{center}
		$\sum_{i=0}^{n} \sum_{j=i}^{n}(i+j)$
	\end{center} 
	\noindent
	\textcolor{blue}{
	$\sum_{i=0}^{n} \sum_{j=i}^{n}(i+j)$\\
	$=\sum_{i=0}^{n} (\sum_{j=i}^{n} i + \sum_{j=i}^{n} j)$\\
	$=\sum_{i=0}^{n}(n-i+1)i+(\sum_{j=1}^{n}j - \sum_{j=1}^{i}j)$\\
	$=\sum_{i=0}^{n}(ni-i^2+i)+(\dfrac{1}{2}n(n+1))-\dfrac{1}{2}i(i+1)$\\
	$=\sum_{i=0}^{n}(ni-i^2+i)+\sum_{i=0}^{n}(\dfrac{1}{2}n(n+1))-\sum_{i=0}^{n}\dfrac{1}{2}i(i+1)$\\
	$=\sum_{i=0}^{n}(ni-i^2+i)+(n+1)(\dfrac{1}{2}n(n+1))-\dfrac{1}{2}((n+1)(\dfrac{1}{2}n+\dfrac{1}{6}n(n+1)(2n+1)))$\\
	$=n\sum_{i=0}^{n}i-\sum_{i=0}^{n}i^2+\sum_{i=0}^{n}i+(n+1)(\dfrac{1}{2}n(n+1))--\dfrac{1}{2}((n+1)(\dfrac{1}{2}n+\dfrac{1}{6}n(n+1)(2n+1))$\\
	$=\dfrac{1}{2}n(n+1)(n+2)$\\
	}
	
	\noindent9.3(l)
	\begin{center}
		$\sum_{i=1}^{n} \sum_{j=1}^{n}\ln{(ij)}$
	\end{center} 
	\noindent
	\textcolor{blue}{
		$\sum_{i=1}^{n} \sum_{j=1}^{n}\ln{(ij)}$\\
		$=\sum_{i=1}^{n} (\sum_{j=1}^{n}\ln{(i)}+\sum_{j=1}^{n}ln(j))$\\
		$=\sum_{i=1}^{n} (n \ln(i) + \ln(j)!)$\\
		$=n\ln(i)!+n\ln(j)!$\\
		$=n(\ln(i)!+\ln(j)!)$
	}
\end{addmargin}
%Q 9.3 ends here

\newpage
%Q 10.9 starts from here
\begin{addmargin}[2em]{2em}
	Problem 10.9: How many zeros are at the end of $1000!$ ?\\
	\textcolor{blue}{
	\indent This is equivalent to finding n such that $10^n|1000!$. Numbers in the end only depends on 5 since it is the greatest in the prime pair for 10. Therefore the numbers in the end are given by $\left \lfloor\dfrac{1000!}{5^1}\right \rfloor+\left \lfloor\dfrac{1000!}{5^2}\right \rfloor+\left \lfloor\dfrac{1000!}{5^3}\right \rfloor+\left \lfloor\dfrac{1000!}{5^4}\right \rfloor+...= 200+48 +1 +0+...=249$. Therefore there are $249$ zeros in the end of $1000!$}
\end{addmargin}
%Q 10.9 ends here

\newpage
%Q 10.41 starts from here
\begin{addmargin}[2em]{2em}
	Problem 10.41(b)(iii). what is the last digit of $2^{70} + 3^{70}$\\
	\indent
	\textcolor{blue}{$2^{70}$ can be re-written as $(2^2)^{35}$. Therefore the last digit of $2^{70}$ is 4. Likewise, $3^{70}$ can be re-written as $(3^2)^{35}$. Therefore the last digit of $3^{70}$ is 9. Summing them up gives us $13$. Therefore the last digit is $3$.}
\end{addmargin}
%Q 10.41(iii)(d) ends here
\end{document}