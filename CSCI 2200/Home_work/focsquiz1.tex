\documentclass{article}
\title{CSCI 2200 HW1}
\author{Xinshi,Wang}
\usepackage[letterpaper,textwidth=5.5in,right=0.6in,textheight=9in,left=0.6in,top=0.7in,bottom=0.7in]{geometry}
\usepackage{scrextend}
\usepackage{graphicx}
\usepackage{xcolor}
\usepackage{amssymb}
\usepackage{amsmath}
\usepackage{setspace}
\begin{document}
\noindent
FOCS Quiz 1 \\
Wang Xinshi\\

Exercise 1

(a) We will prove the contrapositive.

Suppose $I - T$ is not invertible. There exists non-zero $v \in V$ such that $Iv - Tv = 0$, which implies $Tv = v$. Applying $T$ to both sides of this last equation, yields $TTv = Tv = v$. Continuing this, we see that for any positive integer $n$, $T^n v = v$. Therefore $T^n$ can never be $0$.

Now suppose $T^n = 0$. We have

$$ \begin{aligned} (I - T)(I + T + \dots + T^{n-1}) &= (I - T) \sum\limits_{k=0}^{n-1} T^k\\ &= \sum\limits_{k=0}^{n-1} T^k - \sum\limits_{k=0}^{n-1} T^{k + 1}\\ &= I + \sum\limits_{k=1}^{n-1} T^k - \sum\limits_{k=1}^n T^k\\ &= I + \sum\limits_{k=1}^{n-1} T^k - T^n - \sum\limits_{k=1}^{n-1} T^k\\ &= I - T^n\\ &= I\\ \end{aligned} $$

Therefore $I - T$ is an inverse of $I + T + \dots + T^{n-1}$, which implies the desired result.

(b) I wouldn't.

Exercise 2

Suppose $\lambda$ is an eigenvalue of $T$ and $v$ a corresponding eigenvector. Then

$$ \begin{aligned} 0 &= (T - 2I)(T - 3I)(T - 4I)v\\ &= (T^3 - 9T^2 + 26T - 24I)v\\ &= T^3 v - 9T^2 v + 26T - 24v\\ &= \lambda^3 v - 9\lambda^2 v + 26\lambda v - 24v\\ &= (\lambda^3 - 9\lambda^2 + 26\lambda - 24)v\\ \end{aligned} $$

Therefore, because $v \neq 0$, $\lambda^3 - 9\lambda^2 + 26\lambda + 24 = 0$. But we can factor this to $(\lambda - 2)(\lambda - 3)(\lambda - 4) = 0$. Thus $\lambda$ is either $2$, $3$ or $4$.

Exercise 3

Let $v \in V$. Then

$$ 0 = (T^2 - I^2)v = (T + I)(T - I)v $$

which implies $(T - I)v \in \operatorname{null} (T + I)$. We but need to prove that $T + I$ is injective and we will have $Tv = v$. Let $u, w \in V$ such that $(T + I)u = (T + I)w$. Then $Tu + u = Tw + w$ which implies $T(u - w) = w - u = -1(u - w)$. But $-1$ is not an eigenvalue of $T$, thus $u - w = 0$, implying $u = w$, as desired.

Exercise 4

Suppose $v \in V$. We have $v = (I - P)v + Pv$. Because $P - P^2 = 0$, it follows that $P(I - P)v = 0$. Therefore $(I - P)v \in \operatorname{null} P$. Obviously $Pv \in \operatorname{range} P$, thus $v \in \operatorname{null} P + \operatorname{range} P$ and we have $V \subset \operatorname{null} P + \operatorname{range} P$. The inclusion in the opposite direction is clearly true, therefore $V = \operatorname{null} P + \operatorname{range} P$.

To see why this is a direct sum, suppose $u \in \operatorname{null} P \cap \operatorname{range} P$. Because $u \in \operatorname{range} P$, there exists $w$ such that $Pw = u$. But $u \in \operatorname{null} P$, so we must have

$$ 0 = Pu = P^2 w = Pw = u $$

Therefore $\operatorname{null} P \cap \operatorname{range} P = \{0\}$ and by 1.45 it is direct sum.

Exercise 5

Note that, for non-negative $k$, $(STS^{-1})^k = ST^kS^{-1}$, because the $S$'s and $S^{-1}$'s cancel out (this can easily me proven by induction on $k$). Let $n = \deg p$ and $a_0, a_1, \dots, a_n \in \mathbb{F}$ be the coefficients of $p$. Then

$$ \begin{aligned} p(STS^{-1}) &= \sum\limits_{k=0}^n a_k(STS^{-1})^k\\ &= \sum\limits_{k=0}^n a_k S T^k S^{-1}\\ &= S(\sum\limits_{k=0}^n a_k T^k)S^{-1}\\ &= Sp(T)S^{-1}\\ \end{aligned} $$

Exercise 6

It is easy to see that $T^k u \in U$ for $u \in U$ and non-negative $k$ (this can easily be proven by induction on $k$). Since $p(T)u$ is a sum of terms like this, multiplied by the coefficients of $p$, and $U$ is closed under addition and scalar multiplication, then $p(T)u \in U$.

Exercise 7

Suppose $9$ is an eigenvalue of $T^2$. Let $v$ be a corresponding eigenvector. Then $(T^2 - 9I)v = 0$, which implies $(T - 3I)(T + 3I)v = 0$. If $(T + 3I)v = 0$ then $-3$ is an eigenvalue of $T$. Otherwise, $(T + 3I)v$ is an eigenvector of $T$ and $3$ is a corresponding eigenvalue.

Conversely, suppose $\pm 3$ is an eigenvalue of $T$. Let $v$ be a corresponding eigenvector. Then $T^2 v = T(\pm 3v) = (\pm 3)^2 v = 9v$, showing that $9$ is an eigenvalue of $T^2$.

Exercise 8

Define $T \in \mathcal{L}(\mathbb{R}^2)$ by

$$ T(x, y) = \frac{1}{\sqrt{2}}(x - y, x + y) $$

Then, for $(a, b) \in \mathbb{R}^2$, we have

$$ \begin{aligned} T^4 (a, b) &= \frac{1}{\sqrt{2}} T^3 (a - b, a + b)\\ &= \frac{1}{2} T^2 (-2b, 2a)\\ &= \frac{1}{2\sqrt{2}} T (-2a - 2b, 2a - 2b)\\ &= \frac{1}{4}(-4a, -4b)\\ &= -(a, b)\\ \end{aligned} $$

Exercise 9

Let $c(z - \lambda_n) \dotsm (z - \lambda_1)$ be a factorization of $p$. Then $c(T - \lambda_n I) \dotsm (T - \lambda_1 I)$ is a factorization of $p(T)$. Since $p$ is the polynomial of smallest degree such that $p(T)v = 0$, it follows that $(T - \lambda_j I) \dotsm (T - \lambda_1 I)v \neq 0$, for $j < n$. Therefore, we have that $(T - \lambda_{n - 1} I) \dotsm (T - \lambda_1 I)v \neq 0$ is an eigenvector of $T$ and $\lambda_n$ the corresponding eigenvalue. Note that, by 5.20, the order of factorization can be changed, placing any other factor $(T - \lambda_j)$ in the beginning. This implies that all $\lambda$'s are indeed eigenvalues of $T$.

Exercise 10

Note that $T^k v = \lambda^k v$. Let $n = \deg p$ and $a_0, a_1, \dots, a_n \in \mathbb{F}$ be the coefficients of $p$. Then

$$ \begin{aligned} p(T)v &= (\sum\limits_{k=0}^n a_k T^k)v\\ &= \sum\limits_{k=0}^n a_k T^k v\\ &= \sum\limits_{k=0}^n a_k \lambda^k v\\ &= (\sum\limits_{k=0}^n a_k \lambda^k) v\\ &= p(\lambda) v\\ \end{aligned} $$

Exercise 11

Suppose $\alpha$ is an eigenvalue of $p(T)$. Let $c(z - \lambda_1) \dotsm (z - \lambda_n)$ be a factorization of $p(z) - \alpha$. We have

$$ p(T) - \alpha I = c(T - \lambda_1 I) \dotsm (T - \lambda_n I) $$

Because $p(T) - \alpha I$ is not injective, it follows that, for some $j$, $T - \lambda_j I$ is not injective. Therefore $\lambda_j$ is an eigenvalue of $T$. Since $\lambda_j$ is a root of $p(z) - \alpha$, we have that $p(\lambda_j) = \alpha$.

The converse is the same as Exercise 10.

Exercise 12

Define $T \in \mathcal{L}(\mathbb{R}^4)$ by

$$ T(x_1, x_2, x_3, x_4) = (x_2, x_3, x_4, -x_1) $$

Let $p \in \mathcal{P}(R)$ such that $p(x) = x^4$. Then $-1$ is an eigenvalue of $p(T)$, but $p$ is always positive, therefore no eigenvalue $\lambda$ of $T$ satisfies $p(\lambda) = -1$.

Exercise 13

By 5.21, $W$ is either $\{0\}$ or infinite-dimensional. Let $U$ be a subspace of $W$ invariant under $T$. Then $T\rvert_U$ also has no eigenvalues. But $T\rvert_U$ is also an operator on a complex vector space, therefore $U$ is either $\{0\}$ or infinite-dimensional.

Exercise 14

Suppose $V$ is finite-dimensional vector space. Let $v_1, \dots, v_n$ be a basis of $V$. Define $T \in \mathcal{L}(V)$ by

$$ \begin{aligned} Tv_j &= v_{j+1}, \text{ for } j = 1, \dots, n - 1\\ Tv_n &= v_1\\ \end{aligned} $$

$T$ is clearly invertible, but $\mathcal{M}(T)$ with respect to same basis only has zeros in the diagonal.

Exercise 15

Suppose $V$ is finite-dimensional vector space. Let $v_1, \dots, v_n$ be a basis of $V$. Define $T \in \mathcal{L}(V)$ by

$$ Tv_j = v_1 + \dots + v_n $$

$\mathcal{M}(T)$ contains $1$'s in all its entries, but $T$ is clearly not inveritible.

Exercise 16

Let $n = \operatorname{dim} V$. Define $\Psi \in \mathcal{L}(\mathcal{P}_n(\mathbb{C}), V)$ by

$$ \Psi(p) = (p(T))v $$\label{key}

for $p \in \mathcal{P}_n(\mathbb{C})$. One can easily verify that $\Psi$ is linear. Since $\operatorname{dim} \mathcal{P}_n(\mathbb{C}) > \operatorname{dim} V$, by 3.23, there exists $p \in \mathcal{P}_n(\mathbb{C})$ such that $0 = \Psi(p) = (p(T))v$. The rest follows exactly as 5.21.

Exercise 17

This is almost the same as Exercise 16.

Exercise 18

Note that $f$ can only output integer values. Thus, if $f$ is not constant, there will be a jump discontinuity at some point. We will prove $f$ is not constant.

If $T$ is invertible, then the existence of an eigenvalue of $T$ (guaranteed by 5.21) implies that $T - \lambda I$ is not surjective for some $\lambda \in \mathbb{F}$. Hence $f(0) = \operatorname{dim} \operatorname{range} T > \operatorname{dim} \operatorname{range} (T - \lambda I) = f(\lambda)$.

If $T$ is not invertible, choose $\lambda$ such that it is not an eigenvalue of $T$. Then, for any non-zero $v \in V$, $(T - \lambda I)v \neq 0$, showing that $T - \lambda I$ is injective and, therefore, surjective. Hence $f(0) = \operatorname{dim} \operatorname{range} T < \operatorname{dim} \operatorname{range} (T - \lambda I) = f(\lambda)$.

Exercise 19

5.20 implies that any two operators in $\{p(T): p \in \mathcal{P}(\mathbb{F})\}$ commute. But this is obviously not true for $\mathcal{L}(V)$, because $\operatorname{dim} V > 1$. For example, let $v_1, \dots, v_n$ be a basis of $V$. Define $S, R \in L(V)$ by

$$ \begin{aligned} Sv_1 &= v_2, Sv_j = 0 \text{ for } j = 2, \dots, n\\ Rv_1 &= 0, Rv_j = v_j \text{ for } j = 2, \dots, n. \end{aligned} $$

Then $SRv_1 = 0$ but $RSv_1 = v_2$. Thus $SR \neq RS$.

Exercise 20

This follows directly from 5.27 and 5.26.
$\hat{y_i}=0$
\end{document} 