\documentclass[12pt]{article}
\usepackage[letterpaper,textwidth=5.5in,right=0.6in,textheight=9in,left=0.6in,top=0.7in,bottom=0.7in]{geometry}

\usepackage{amsmath,amssymb}
\usepackage{amsmath}
\usepackage{algorithm}
\usepackage[noend]{algpseudocode}
\makeatletter
\def\BState{\State\hskip-\ALG@thistlm}
\begin{document}
	\noindent Xinshi Wang\\
	661975305\\
	HW02\\
	
	\noindent Problem 1\\
	
	1. Loop Invariant: $y>=0 \land result + xy = mn$\\
	
	2. Base Case: $result = 0 \land x = m \land y = n \implies result + xy = mn$.\\
	$y = 0 \implies y \geq 0$
	
	3. Assume $result + xy = mn$ holds at $k$ iteration, and prove $result + xy = mn$ holds at $k+1$ iteration. Consider two exhaustive cases:\\
	
	a. If $y$ is even:
	
	$x_1 = x + x \land y_1 = y/2 \implies y_1 \geq 0 \land result + xy = mn \implies result_1 + xy = result + (x+x)(y/2) = result + xy = mn$
	
	Thus the Inductive hypothesis holds at the case when $y$ is even.\\
	
	b. Else(The odd case):
	
	$result_1 = result + x \land y_1 = y - 1 \geq 0 \implies result_1 + xy_1 = result + x + xy_1$

	 $= result + x + x(y-1) = result + x + xy - x = result + xy = mn$
	 
	Thus the Inductive hypothesis holds at the case when $y$ is odd.\\
	
	4. At exit: $!(y != 0) \land result + xy = mn$

		$\implies y = 0 \land result + xy = mn$ 
		
		$\implies result + 0 = mn$
		
		$\implies result = mn$\\
	
	5. $D = y$
	
	When $y$ is even: $y_1 = y/2 \implies y_1 < y$. Thus $D$ decreases.
	
	When $y$ is odd: $y_1 = y - 1 \implies y_1 < y$. Thus $D$ decreases.
	
	Since $D = y$ decreases in both cases. $D$ is decrementing in every iteration.
	
	When $D$ reaches its minumum $0$, $y = 0$. When $y = 0$, the loop exits.\\
	
	\newpage
\noindent Problem2

	a.
	
	Input: A directed non-empty array with "blue" and "red"
	
	Output: A sorted array with $0~k-1$ being red and $k~N-1$ being blue.
	\begin{algorithm}
		\caption{Dutchflag}\label{dutchflag}
		\begin{algorithmic}[1]
			\Procedure{dutchflag}{arr}
			\State $i = 0$
			\State $j = len(arr) - 1$
			\While{$i<=j$}
			\If{$arr[i] == "blue"$ and $arr[j] == "red"$}
			\State $swap(i,j,arr)$
			\State $i += 1$
			\State $j -= 1$
			\EndIf
			\If{$arr[i] == "red"$ and $arr[j] == "blue"$}
			\State $i += 1$
			\State $j -= 1$
			\EndIf
			\If{$arr[i] == "red" \land arr[j] == "red"$}
			\State $i += 1$
			\EndIf
			\If{$arr[i] == "blue" \land arr[j] == "blue"$}
			\State $j -= 1$
			\EndIf
			\EndWhile
			\State $k = j+1$
			\EndProcedure
		\end{algorithmic}
	\end{algorithm}
	
	b.
	
	\noindent $0<=k<=arr.Length$\\
	$forall e:: 0<=e<k ==> arr[e] == "red"$\\
	$forall f:: k<=f<arr.Length ==> arr[f] == "blue"$\\
	
	c.
	
	\noindent $forall d:: 0<=d<=arr.Length-1 ==> arr[d] == 'b' || arr[d] == 'r'$\\
	$0<=i<=arr.Length$\\
	$-1<=j<=arr.Length-1$\\
	$forall b:: 0<=b<i ==> arr[b] == 'r'$\\ 
	$forall c:: j<c<arr.Length ==> arr[c] == 'b'$\\
	
	\newpage
	
	\noindent Problem 3
	
	Inner LI: $1 \leq s \leq r+1 \land u == v * s$
	
	Base Case:
	
	$r = 0 ,s = 1, u = 1, v = u \implies 1 \leq s \leq r+1$
	
	In order to show $u == v * s$ hold at the base case, we substitue the values in and get $1 == 1 * 1$.
	
	Inductive step: Assume the Inner LI holds for the base case, prove it holds at the $k+1$ iteration.
	
	$r_k = 0$
	
	$v_k = u_k$
	
	$s_{k+1} = s_k + 1$
	
	$u_{k+1} = u_k + v_k$
	
	
	$1 \leq s_k \implies 1 \leq s_k+1 \implies 1 \leq s_k + 1 \leq r_k + 1 \implies 1 \leq s_{k+1} \leq r_{k+1}$
	
	$u_{k+1} = u_k + v_k \land u_k = v_k * s_k \implies v_{k+1} = v_k * (s_k + 1) = v_k(s_k) + v_k$
	
	$u_k = u_{k+1} - v_k = v_k(s_k) + v_k - v_k$
	
	$u_k = v_k * s_k $
	
	Thus the inductive steps hold.\\\\
	
	Outer LI: $0 \leq r \leq n \land u == Factorial(r)$
	
	Base Case:
	
	$u = 1 \land  r = 0 \implies u = Factorial(r)$
	
	$r = 0 \land n \geq 0 \implies 0 \leq r \leq n$
	
	Inductive step: Assume the Inner LI holds for the base case, prove it holds at the $k+1$ iteration.
	
	$u_k+1 = u_k * v_k$
	
	$v_{new} = u_{old}$
	
	$r_k+1 = r_k + 1$
	
	$0 \leq r_k + 1 \leq n \implies 0 \leq r_k+1 \leq n$
	
	$u_{new} = v*s = v*(r_{old}+1) = Factorial(r_old) * r_{old}+1 = Factorial(r_old+1)$	
	
	Thus the inductive steps hold.
 \end{document}