\documentclass{article}
\usepackage{fullpage}
\usepackage{ulem}
\usepackage{amsmath,amssymb}
\pagestyle{empty}
\newcommand{\D}{\displaystyle}

\begin{document}
\noindent{\bf Homework 4}

There are three homework problems. The expectation is that students
will submit very high quality proofs. The problem solutions must be written
in LaTeX and compiled to a pdf file. 
Follow the instructions on LMS for how to submit the problem.

\bigskip
\bigskip


\begin{enumerate}

\item Prove that  $14^n - 5^n$ is a multiple of 9 for all $n \in \mathbb{N}$. {\it hint: show that $14^n - 5^n = 9 m$ for some integer $m$.}
\medskip
\vspace{.15in}

\noindent {\bf Proof.} Suppose $n \in \mathbb{N}$. Let $P(n)$ be the statement $14^n-5^n$ is a multiple of 9 for all $n \in \mathbb{N}$. In other words there exists an $m \in \mathbb{Z}$ such that $14^n-5^n = 9m$ for all $n \in \mathbb{N}$. We prove this by mathematical induction.\\\\
We first show that $P(1)$ is true by substituting $n = 1$ into the LHS formula and showing it is equivalent to the RHS formula.

\begin{center}
	$14^1 - 5^1 = 9 = 9(1)$
\end{center}  
Therefore we have shown the LHS is equivalent to RHS when $n = 1$.\\\\
Next we let $k \in \mathbb{N}$ and assume $P(k):14^k-5^k=9m,m \in \mathbb{Z}$ is true and try to prove $P(k+1):14^{k+1}-5^{k+1}=9m,m \in \mathbb{Z}$ is true. This follows by adding $13(14^n)-4(5^n)$ to both sides of $P(k)$ and simplify to get $P(k+1)$.

\begin{align*}
	14^n-5^n+13(14^n)-4(5^n) &= 9m+13(14^n)-4(5^n)\\
	14^{n+1}-5^{n+1}&=9m+13(14^n)-4(5^n)\\
	\because 14^{k}-5^{k}&=9m\\
	\therefore 14^k&=9m+5^k\\
	14^{n+1}-5^{n+1}&=9m+13(14^n)-4(5^n)=9m+13(9m+5^k)-4(5^n)\\
	14^{n+1}-5^{n+1}&=9m+13(9m)-4(5^n)+13(5^n)-4(5^n)\\
	14^{n+1}-5^{n+1}&=9m+9(13m)+9(5^n)\\
	Let \quad m_1 \quad be \quad &m+13m+5^n,\\
	14^{n+1}-5^{n+1} &= 9m_1,m_1 \in \mathbb{Z}.
\end{align*}
Therefore we have shown $P(1)$ is true and if $P(k)$ is true $P(k+1)$ is true. Thus we have proved $P(n)$ is true.

\item {\it typo fixed at 11 PM on 10/6**} Prove that $\D \left(1-\frac{1}{2^2} \right) \left(1-\frac{1}{3^2} \right)  \left(1-\frac{1}{4^2} \right)  \cdots  \left(1-\frac{1}{n^2} \right) = \frac{n+1}{2n}$ for all $n \in \mathbb{N}$ with $n \geq 2$. 
\medskip
\vspace{.15in}

\noindent {\bf Proof.} Suppose $n \in \mathbb{N}$. Let $P(n)$ be the statement $\D \left(1-\frac{1}{2^2} \right) \left(1-\frac{1}{3^2} \right)  \left(1-\frac{1}{4^2} \right)  \cdots  \left(1-\frac{1}{n^2} \right) = \frac{n+1}{2n}$ for all $n \in \mathbb{N}$ with $n \geq 2$. We prove this by mathematical induction.\\\\
We first show that $P(2)$ is true by substituting $n = 2$ into the LHS formula and showing it is equivalent to the RHS formula.

\begin{center}
	$\D \left(1-\frac{1}{2^2} \right) = \dfrac{3}{4} = \dfrac{2+1}{2 \times 2}$
\end{center}  
Therefore we have shown the LHS is equivalent to RHS when $n = 2$.\\\\
Next we let $k \in \mathbb{N}$ and assume $P(k):\D \left(1-\frac{1}{2^2} \right) \left(1-\frac{1}{3^2} \right)  \left(1-\frac{1}{4^2} \right)  \cdots  \left(1-\frac{1}{k^2} \right) = \frac{k+1}{2k}$ is true and try to prove $P(k+1):\D \left(1-\frac{1}{2^2} \right) \left(1-\frac{1}{3^2} \right)  \left(1-\frac{1}{4^2} \right)  \cdots  \left(1-\frac{1}{(k+1)^2} \right) = \frac{(k+1)+1}{2(k+1)}$ is true. This follows by multiplying $\D \left(1-\frac{1}{(k+1)^2} \right)$ to both sides of $P(k)$ and simplify to get $P(k+1)$.

\begin{align*}
	\D \left(1-\frac{1}{2^2} \right) \left(1-\frac{1}{3^2} \right)  \left(1-\frac{1}{4^2} \right)  \cdots  \left(1-\frac{1}{(k+1)^2} \right) &= \frac{k+1}{2k} \times \D \left(1-\frac{1}{(k+1)^2} \right)\\
	\D \left(1-\frac{1}{2^2} \right) \left(1-\frac{1}{3^2} \right)  \left(1-\frac{1}{4^2} \right)  \cdots  \left(1-\frac{1}{(k+1)^2} \right) &= \frac{k+1}{2k} - \dfrac{1}{2k} \cdot \dfrac{1}{k+1}\\
	\D \left(1-\frac{1}{2^2} \right) \left(1-\frac{1}{3^2} \right)  \left(1-\frac{1}{4^2} \right)  \cdots  \left(1-\frac{1}{(k+1)^2} \right) &= \dfrac{k^2+2k+1}{2k(k+1)}-\dfrac{1}{2k(k+1)}\\
	\D \left(1-\frac{1}{2^2} \right) \left(1-\frac{1}{3^2} \right)  \left(1-\frac{1}{4^2} \right)  \cdots  \left(1-\frac{1}{(k+1)^2} \right) &= \dfrac{k^2+2k}{2k(k+1)}\\
	\D \left(1-\frac{1}{2^2} \right) \left(1-\frac{1}{3^2} \right)  \left(1-\frac{1}{4^2} \right)  \cdots  \left(1-\frac{1}{(k+1)^2} \right) &= \dfrac{(k+1)+1}{2(k+1)}\\
\end{align*}
Therefore we have shown $P(1)$ is true and if $P(k)$ is true $P(k+1)$ is true. Thus we have proved $P(n)$ is true.
\item Define $\D H_j = 1 + \frac{1}{2} + \frac{1}{3} + \cdots + \frac{1}{j} = \sum_{j=1}^n \frac{1}{j}$~.  Use mathematical induction to prove that $\D H_{2^n} \geq 1 + \frac{n}{2}$ is true for all $n \in \mathbb{N}$. 
\vspace{.15in}

\noindent {\bf Proof.} Suppose $n \in \mathbb{N}$. Let $P(n)$ be the statement $\D 1 + \frac{1}{2} + \frac{1}{3} + \cdots + \frac{1}{2^n} \geq \D 1 + \frac{n}{2}$ for all $n \in \mathbb{N}$. We prove this by mathematical induction.\\\\
We first show that $P(1)$ is true by substituting $n = 1$ into the LHS formula and showing it is equivalent to the RHS formula.

\begin{center}
	$\D 1+\dfrac{1}{2} \geq 1+\dfrac{1}{2}$
\end{center}  
Therefore we have shown the LHS is equivalent to RHS when $n = 1$.\\\\
Next we let $k \in \mathbb{N}$ and assume $P(k):\D 1 + \frac{1}{2} + \frac{1}{3} + \cdots + \frac{1}{2^k} \geq \D 1 + \frac{k}{2}$ is true and try to prove $P(k+1):\D 1 + \frac{1}{2} + \frac{1}{3} + \cdots + \frac{1}{2^{k+1}} \geq \D 1 + \frac{(k+1)}{2}$ is true. This follows by adding $\D \frac{1}{2^{k+1}}$ to both sides of $P(k)$ and simplify to get $P(k+1)$.

\begin{align*}
	\D 1 + \frac{1}{2} + \frac{1}{3} + \cdots + \frac{1}{2^k} + \frac{1}{2^{k+1}} &\geq \D 1 + \frac{k}{2} + \frac{1}{2^{k+1}}\\
	\because \D \frac{1}{2^{k+1}} &\leq \D \frac{1}{2},\forall k \in \mathbb{N}\\
	\therefore \D 1 + \frac{1}{2} + \frac{1}{3} + \cdots + \frac{1}{2^k} + \frac{1}{2^{k+1}} &\geq \D 1 + \frac{k}{2} + \frac{1}{2}\\
	\D 1 + \frac{1}{2} + \frac{1}{3} + \cdots + \frac{1}{2^k} + \frac{1}{2^{k+1}} &\geq \D 1 + \frac{k+1}{2}\\
\end{align*}
Therefore we have shown $P(1)$ is true and if $P(k)$ is true $P(k+1)$ is true. Thus we have proved $P(n)$ is true.
\end{enumerate}

\bigskip
\bigskip
Note: Do not cite axioms of real numbers (sec 3.2) on each step of your proof.

\end{document}
