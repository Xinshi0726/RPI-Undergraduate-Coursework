\documentclass{article}
\usepackage{amsmath,amssymb,fullpage}
\usepackage{scrextend}

\pagestyle{empty}

\begin{document}
	\noindent{\bf Problem Presentation 3}
	
	For each set and metric space combination listed below, determine if the set is compact or not. For each set
	that is compact, explain how you know it is compact. For each set that is not compact, describe an open
	cover with no finite subcover.\\
	
	\noindent $\mathbb{R}$ with the usual metric, $S = \{1+\dfrac{1}{n}:n \in \mathbb{N}\}$\\
	
	The set is not compact. We could find an open cover $\mathcal{F} = \{(1+\dfrac{1}{n},3):n \in \mathbb{N}\}$ such that $S \subseteq \bigcup_{A \in \mathcal{F}} A$ and $S$ has no finite subcover. In order to show $\mathcal{F}$ covers $S$, let $x \in S$, we have $1<x\leq 2$. Thus $1<x\leq 3$ holds and therefore $x \in \mathcal{F}$. Hence $S \subseteq \bigcup{A \in \mathcal{F}} A$. Since $\mathcal{F}$ has open intervals, it is open. Therefore, we know $\mathcal{F}$ is an open cover of $S$. A subcollection of $\mathcal{F}$ has the form $\mathcal{G} = \{(1+\dfrac{1}{M},3):M=1,2,...,n \land n<|\mathbb{N}|\}$. We can find a number $1+\dfrac{1}{n+1} \in S$ but $\notin \mathcal{G}$. Therefore $\mathcal{G}$ does not cover all the points in $S$. Thus $S$ is not compact. \\
	
	\noindent $\mathbb{R}$ with the usual metric, $S = [1, 2] \cup [5, 8]$\\
	
	From the Heine-Borel Theorem, we know a subset $S$ of $\mathbb{R}$ with the Euclidean metric is compact if $S$ is (i). closed and (ii). bounded.\\
	
	In order to prove $S$ is closed, we need to prove $X \setminus S$ is open. Thus we need to find an $\epsilon > 0$ such that $N(x;\epsilon) \subseteq X \setminus S$. Let $x \in X \setminus ([1,2] \cup [5,8])$ be given. Thus $x \in X \setminus [1,2]$ or $x \in X \setminus [5,8]$. Consider two exhaustive cases: (i). $x \in X \setminus [1,2]$ (ii). $x \in X \setminus [5,8]$.\\
	
	First we prove $S$ is closed. Let $x \in X \setminus [1,2]$ be given. Pick $\epsilon$ = $\min \{|x-1|,|x-2|\}$. We argue that $N(x;\epsilon) \subseteq X \setminus ([1,2] \cup [5,8])$. By definition we know $N(x;\epsilon) = \{y \in \mathbb{R}:|x-y|<\epsilon\}$. Let $y \in N(x;\epsilon)$, then we have $|y-x|<\epsilon$.Thus we know
	\begin{align*}
		-\epsilon<&y-x<\epsilon\\
		x-\epsilon<&y<\epsilon+x
	\end{align*}
	Since $x \in X \setminus [1,2]$, $x \in X$ and $x \notin [1,2]$. Thus $x \leq 1$ or $x \geq 2$. Consider two exhaustive cases: (i). $x \leq 1$ (ii). $x \geq 2$. If $x \leq 1$, then $\epsilon = |x-1| = 1-x$. Thus $1-\epsilon<y<1-x+x=1$. Therefore $1-\epsilon<y<1$ so $1-\epsilon<y \leq 1$ holds. Hence we can conclude $y \notin [1,2]$ and $y \in X \setminus [1,2]$. If $x \geq 2$, $\epsilon = |x-2| = x-2$. Thus we have $x-x+2<y<\epsilon+2$. Thus $2<y<\epsilon+2$ which implies $2<y \leq \epsilon+2$. Hence we can conclude $y \notin [1,2]$ and $y \in X \setminus [1,2]$. Since both cases holds, we can conclude $N(x;\epsilon) \subseteq X \setminus [1,2]$, which implies $N(x;\epsilon) \subseteq \setminus ([1,2] \cup [5,8])$.\\
	
	Let $x \in X \setminus [5,8]$ be given. Pick $\epsilon$ = $\min \{|x-5|,|x-8|\}$. We argue that $N(x;\epsilon) \subseteq X \setminus ([1,2] \cup [5,8])$. By definition we know $N(x;\epsilon) = \{y \in \mathbb{R}:|x-y|<\epsilon\}$. Let $y \in N(x;\epsilon)$, then we have $|y-x|<\epsilon$.Thus we know
	\begin{align*}
		-\epsilon<&y-x<\epsilon\\
		x-\epsilon<&y<\epsilon+x
	\end{align*}
	Since $x \in X \setminus [5,8]$, $x \in X$ and $x \notin[5,8]$. Thus $x \leq 5$ or $x \geq 8$. Consider two exhaustive cases: (i). $x \leq 5$ (ii). $x \geq 8$. If $x \leq 5$, then $\epsilon = |x-5| = 5-x$. Thus $5-\epsilon<y<5-x+x=5$. Therefore $5-\epsilon<y<5$ so $5-\epsilon<y \leq 5$ holds. Hence we can conclude $y \notin [5,8]$ and $y \in X \setminus [5,8]$. If $x \geq 8$, $\epsilon = |x-8| = x-8$. Thus we have $x-x+8<y<\epsilon+8$. Thus $8<y<\epsilon+8$ which implies $8<y \leq \epsilon+8$. Hence we can conclude $y \notin [5,8]$ and $y \in X \setminus [5,8]$. Since both cases holds, we can conclude $N(x;\epsilon) \subseteq X \setminus [5,8]$, which implies $N(x;\epsilon) \subseteq \setminus ([1,2] \cup [5,8])$.\\
	
	Since our choice of $x$ is arbitrary, we have proved set $S$ is closed.\\
	
	Next we prove $S$ is bounded. In order to prove $S$ is bounded, we need to show $S$ is bounded above and bounded below.\\
	
	First we show $S$ is bounded above. Let $s = 9$ be given. We know $s>2$ and $s>8$. Thus $s>x$ for all $x \in [1,2]$ and $s>x$ for all $x \in [5,8]$, which means $s>x$ for all $x \in [1,2] \cup [5,8]$ and $s$ is a lower bound for S. Since our choice of $x$ is arbitrary, we can conclude $x$ is bounded above.\\ 
	
	Then we show $S$ is bounded below. Let $s = 0$ be given. We know $s<1$ and $s<5$. Thus $s<x$ for all $x \in [1,2]$ and $s>x$ for all $x \in [5,8]$, which means $s<x$ for all $x \in [1,2] \cup [5,8]$ and $s$ is a lower bound for S. Since our choice of $x$ is arbitrary, we can conclude $x$ is bounded below.\\ 
	
	Since our choice of $x$ is arbitray, we can conclude $x$ is bounded above.\\
	
	Since $S$ is bounded and closed in $\mathbb{R}$ with the usual metric, we can conclude $S$ is compact according to the Heine-Borel Theorem.\\
	
	\noindent $\mathbb{R}$ with the usual metric, $S = [0, 2] \setminus \mathbb{Q}$\\
	
	The set is not compact. We know $1 \in [0,2]$ and $1 \in \mathbb{Q}$. Thus $1 \notin [0,2] \setminus \mathbb{Q}$. Therefore we can reform the set as $\{[0,1-\dfrac{1}{n}) \cup (1,2]:n \in \mathbb{N}\}$. Thus we can cover $S$ with $\mathcal{F} = \{(-1,1-\dfrac{1}{n}) \cup (1,3): n \in \mathbb{N}\}$ since $(-1,1-\dfrac{1}{n})$ covers $[0,1-\dfrac{1}{n})$ and $(1,3)$ covers $(1,2]$. $\mathcal{F}$ is open since it consists of only open sets. Therefore $\mathcal{F}$ is an open cover of $S$. A subcollection of $\mathcal{F}$ has the form $\mathcal{G} = \{(-1,1-\dfrac{1}{M}) \cup (1,3)\:M=1,2,...,n \land n<|\mathbb{N}|\}$. We can find a irrational number q between $1-\dfrac{1}{n}$ and $1-\dfrac{1}{n+1}$ and $q \in S$ but $q \notin \mathcal{G}$. Therefore $\mathcal{G}$ does not cover all the points in $S$. Thus $S$ is not compact. \\
	
    \noindent $\mathbb{R}$ with the discrete metric, $S = [-1, 2] \cup [3, 5]$\\
	
	The set is not compact. Let $S = [-1, 2] \cup [3, 5]$ be given. We know $S$ is infinite since there are infinitely many points in $S$. Let $\mathcal{F} = \{x:x \in S\}$. We know $\mathcal{F}$ is open since singletons are open. $\mathcal{F}$ covers $S$ since $\bigcup_{A \in \mathcal{F}} A = S$ and $S \subseteq S$. Every sub collection of $\mathcal{F}$ has the form $\mathcal{G}=\{{x_i}:x_i \in S \land i = 1,2,3,...,n,n \in \mathbb{N},n<|S|\}$. $\mathcal{G}$ does not cover $S$ since $\{x_{n+1}\}$ is not an element of $\mathcal{G}$, $\mathcal{G}$ does not cover $S$. Thus $S = [-1, 2] \cup [3, 5]$ is not compact.\\
	
	Presentation Link:\\
	https://rensselaer.webex.com/rensselaer/ldr.php?RCID=67ba134696feb9aff722b482437b202f\\
	
	I used RECGO because you can directly record a video on ipad and edit it.
\end{document}
