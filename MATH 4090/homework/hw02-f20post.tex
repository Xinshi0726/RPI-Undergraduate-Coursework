\documentclass{article}
\usepackage{amsmath,amssymb,fullpage}
\usepackage{scrextend}

\pagestyle{empty}

\begin{document}
\noindent{\bf Homework Set 2}

There are three homework problems. The expectation is that students
will submit a very high quality proof for each problem. The homework set solutions must be written
in LaTeX. Follow the instructions on LMS for how to submit the homework set and what file(s) to submit.


\begin{enumerate}
\bigskip
\item  Let $J$ be a non-empty index set and let $\{A_j:j \in J\}$ be an indexed family of sets and suppose $B$ is a set. 
Prove that $$B  \setminus \bigcap_{j \in J} A_j = \bigcup_{j \in J} \left(B \setminus A_j\right)$$  
\bigskip
\vspace{.15in}
%Note the spacing used here to help set the proof apart from the problem statement.

% Insert solution here
\noindent {\bf Proof.} In order to prove $ B  \setminus \bigcap_{j \in J} A_j = \bigcup_{j \in J} (B \setminus A_j)$ we need to show (i) $B \setminus \bigcap_{j \in J} A_j  \subseteq \bigcup_{j \in J} (B \setminus A_j) $  and (ii) $\bigcup_{j \in J} (B \setminus A_j)  \subseteq B \setminus \bigcap_{j \in J} A_j$. \\

First we need to prove $B \setminus \bigcap_{j \in J} A_j  \subseteq \bigcup_{j \in J} (B \setminus A_j) $. Assume $B \setminus \bigcap_{j \in J} A_j$ is non-empty, else the result follows. Let $x \in B \setminus \bigcap_{j \in J} A_j$ which means $x \in B \text{ and } x \notin \bigcap_{j \in J}A_j$. Thus it is false that $x \in j \text{ for all } j \in J$. Therefore there exist $j \in J$ such that $x \notin A_j$ which implies $x \in B \setminus A_{j}$ for some $j \in J$. So $x \in \bigcup_{j \in J} B \setminus A_j$. Therefore we have shown that $B \setminus \bigcap_{j \in J} A_j  \subseteq \bigcup_{j \in J} (B \setminus A_j) $.\\
                    
Then we need to prove $\bigcup_{j \in J} (B \setminus A_j)  \subseteq B \setminus \bigcap_{j \in J} A_j$. Assume $\bigcup_{j \in J} (B \setminus A_j)$ is non-empty else the result follows. Let $x \in \bigcup_{j \in J} (B \setminus A_j)$ which means $x \in B \text{ and } x \notin A_j \text{ for some } j \in J$. Therefore it is false that for all $j \in J$, $x \in A_j$. Hence, $x \notin \bigcap_{j \in J} A_j$. Then we can conclude that $x \in B \setminus \bigcap_{j \in J} A_j$. Therefore we have shown that  $\bigcup_{j \in J} (B \setminus A_j)  \subseteq B \setminus \bigcap_{j \in J} A_j$.\\

\noindent Since we have shown (i) and (ii) we  have shown that $ B \setminus \bigcap_{j \in J} A_j = \bigcup_{j \in J} (B \setminus A_j)$.\\

\item Prove or give a counterexample that $(A \times C) \setminus (B \times C )= (A \setminus B) \times C$. 
\bigskip
\vspace{.15in}
%Note the spacing used here to help set the proof apart from the problem statement.

\noindent {\bf Proof.} In order to prove $(A \times C) \setminus (B \times C) = (A \setminus B) \times C$, we need to show $(A \times C) \setminus (B \times C) \subseteq (A \setminus B) \times C$ and $(A \setminus B) \times C \subseteq (A \times C) \setminus (B \times C)$.\\

First we need to prove $(A \times C) \setminus (B \times C) \subseteq (A \setminus B) \times C$. Assume $(A \times C) \setminus (B \times C)$ is non-empty else the result follows. Let $(x,y) \in (A \times C) \setminus (B \times C)$. Then we know $(x,y) \in (A \times C) \setminus (B \times C)$ which means $(x,y) \in (A \times C)$ and $(x,y) \notin (B \times C)$. Hence we can conclude (i). $x \in A \text{ and } y \in C$  (ii). $x \notin B \text{ or } y \notin C$. Since we know $y \in C$, $y \notin C$ is false. Hence, $x \notin B$ is true. We know $x \in A \setminus B$ and $y \in c$ which means $(x,y) \in (A \setminus B) \times C$. Therefore, $(A \times C) \setminus (B \times C) \subseteq (A \setminus B) \times C$.\\

Then we need to prove $(A \setminus B) \times C \subseteq (A \times C) \setminus (B \times C)$. Assume $(A \setminus B) \times C$ is non-empty, else the result follows. Let $(x, y) \in (A \setminus B) \times C$, which means $x \in (A \setminus B)$ and $y \in C$. Then we know $x \in A$ and $x \notin B$, and $y \in C$. Therefore $(x, y) \in A \times C$ and $(x, y) \notin (B \times C)$. Hence we can conclude $(x, y) \in (A \times C) \setminus (B \times C)$. Therefore we have proved  $(A \setminus B) \times C \subseteq (A \times C) \setminus (B \times C)$.\\

Since $(A \times C) \setminus (B \times C) \subseteq (A \setminus B) \times C$ and $(A \setminus B) \times C \subseteq (A \times C) \setminus (B \times C)$, we have proved $(A \times C) \setminus (B \times C) = (A \setminus B) \times C$.\\

\bigskip
\item Define a relation $\text{\bf R}$ on the set of all integers $\mathbb{Z}$ by $x \text{\bf R} y $ if and only if $x - y = 3k$ for some integer $k$.  \begin{itemize}
\item Prove that $\text{\bf R}$ satisfy the properties reflexive, symmetric, and transitive on $\mathbb{Z}$ and thus is an equivalence relation on $\mathbb{Z}$.
\item Describe the equivalence class of $E_5$. How many distinct equivalence classes are there for $\text{\bf R}$?
\item Define a new relation $\text{\bf R}_2$ on the set of all integers $\mathbb{Z}$ by $x \text{\bf R}_2 y $ if and only if $x + y = 3k$ for some integer $k$.  Does $\text{\bf R}_2$ satisfy all three properties reflexive, symmetric and transitive? Explain, {\it proof not required}.
\end{itemize}
\bigskip
\vspace{.15in}

$\text{\bf 1.Proof:}$
	\begin{addmargin}[1.5cm]{0cm}
		In order to prove $\text{\bf R}$ is an equivalence class, we need to show $\text{\bf R}$ satisfies all three properties reflexive, symmetric and transitive.\\
		(i).Reflexive: $x \text{\bf R} x$
		\begin{addmargin}[0.5cm]{0cm}
			$x - x = 3k$\\
			$0 = 2k$\\
			When k = 0, this statements holds and $0 \in \mathbb{Z}$.
		\end{addmargin}
	\end{addmargin}
	
	\begin{addmargin}[1.5cm]{0cm}
		(i).Symmetric: $x \text{\bf R} y \implies y \text{\bf R} x$
		\begin{addmargin}[0.5cm]{0cm}
			$x \text{\bf R} y:$ $x - y = 3k_1$\\
			$y - x = - (x - y) = -3k_1 = 3k_2$\\
			Therefore $k_1 = -k_2$. Since $k_1 \in \mathbb{Z}$, $k_2 \in \mathbb{Z}.$\\
			$\therefore y \text{\bf R} x:$ $y - x = 3k_2$ is true when $x \text{\bf R} y$ is true\\
			Hence $x \text{\bf R} y \implies y \text{\bf R} x$\\ 
		\end{addmargin}
	\end{addmargin}
	
	\begin{addmargin}[1.5cm]{0cm}
		(i).Transitive: If $x \text{\bf R} y$ and $y \text{\bf R} x$, then $x \text{\bf R} z$
		\begin{addmargin}[0.5cm]{0cm}
			$x \text{\bf R} y:$ $x - y = 3k_1$\\
			$y \text{\bf R} z:$ $y - z = = 3k_2$\\
			$x-y+y-z = 3k_1+3k_2 = x-z$ \\
			Assume $k_3 = k_1 + k_2$. Since $k_1 \in \mathbb{Z}$ and $k_2 \in \mathbb{Z}$, $3k_3 = 3(k_1+k_2) \in \mathbb{Z}$.\\
			$\therefore x \text{\bf R} z:$ $x - z = 3k_3$\\
			Therefore we have proved if $x \text{\bf R} y$ and $y \text{\bf R} x$, then $x \text{\bf R} z$.
		\end{addmargin}
		Hence we have proved $\text{\bf R}$ is an equivalence class
	\end{addmargin}

\bigskip
\vspace{.15in}

$\text{\bf 2.}$ $E_5 = \{y \in \mathbb{R} : y-5 = 3k, k \in \mathbb{Z}\}$. For all $y$ in real numbers such that $y - 5$ is a multiple of $3$ with some integer $k$. $E_5 = \{...,-1,2,5,8,11,...\}$ and the increment between each numbers in the set is 3. Therefore there should be other two sets that are equivalence class of $\text{\bf R}$, $E_3 = \{...,-3,0,3,6,8,...\}$ and $E_4=\{...,-2,1,3,5,9,...\}$. All the other equivalence classes are identical to any of these three sets.

\bigskip
\vspace{.15in}

$\text{\bf 3.}$ $x+x = x+x$, so $x \text{R} x$. $(x+y) = (y+x)$, so $x \text{R} y$. $x+y = 3k_1$ and $y+z = 3k_2$ however does not imply $x+z= 3k_3$. For example, $x = 2$ , $y = 4$, $z = 5$, so $x+y = 6 = 3_k1$ and $y+z = 9 = 3k_2$. However $x + z = 7 \neq 3k_3$. The only requirement here is the sum of $x$ and $y$ is equal to $3$ times an integer, but there's no restriction on either of them. Therefore $\text{R}$ is not transitive.      
\end{enumerate}

\end{document}
