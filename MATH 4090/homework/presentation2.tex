\documentclass{article}
\usepackage{fullpage}
\usepackage{ulem}
\usepackage{amsmath,amssymb}
\pagestyle{empty}
\newcommand{\D}{\displaystyle}

\begin{document}
\noindent{\bf Presentation 2}\\
Wang Xinshi\\
661975305\\

\noindent Problem statement: Use the formal definition of the limit of a function to prove that $$\lim_{x \to 2} 3x^2+6x-5 = 19$$

\noindent \text{\bf Proof:} Let $\epsilon>0$ be given.\\
we can choose $\delta = \min \{1,\dfrac{\epsilon}{21}\}$.\\
Consider $x \in D$ where $0<|x-2|<\delta$, then we have $0<|x-2|<1$ and $0<|x-2|<\dfrac{\epsilon}{21}$.\\
Since $0<|x-2|<1$, we have $|x+4| = |x-2+6| \leq |x-2|+|6| < 7$ by the triangular inequality.\\
Therefore it follows
\begin{align*}
	|f(x)-L|&=|3x^2+6x-5-19|\\
	&=|3x^2+6x-24|\\
	&=|3(x^2+2x-8)|\\
	&=3|(x^2+2x-8)|\\
	&=3|(x+4)(x-2)|\\
	&<3\times7\delta\\
	&=21\delta\\
	&<\epsilon\\
\end{align*}
Since our choice of $\epsilon$ is arbitrary, we have proved $\lim_{x \to 2} 3x^2+6x-5 = 19$.\\

\noindent The link to the presentation:\\
https://rensselaer.webex.com/rensselaer/ldr.php?RCID=d46e32ce2e4391c47a90dd4840c9d51b
\end{document}
