\documentclass{article}

\title{Linear Algebra HW1}
\author{Xinshi,Wang}
\usepackage[letterpaper,textwidth=5.5in,right=0.6in,textheight=9in,left=0.6in,top=0.7in,bottom=0.7in]{geometry}
\usepackage{scrextend}
\usepackage{amsmath}
\usepackage{xcolor}
\usepackage{indentfirst}
\newcommand\aug{\fboxsep=-\fboxrule\!\!\!\fbox{\strut}\!\!\!}
\begin{document}
%problem 1 starts from here
	\noindent
	Linear algebra HW1\\
	Wang Xinshi\\
	
	\noindent Strang\\
	\begin{addmargin}[3em]{3em}
		P.29 \#3 Solve these three equations for $y_{1}$, $y_{2}$, $y_{3}$ in terms of $c_{1}$, $c_{2}$, $c_{3}$:\\ 
		\begin{addmargin}[10em]{5em}
				 $Sy = c, 
				 \begin{bmatrix}
				 	1 & 0 & 0\\
				 	1 & 1 & 0\\
				 	1 & 1 & 1\\
				 \end{bmatrix}
			 	\begin{bmatrix}
			 		y_1\\ 
			 		y_2\\
			 		y_3\\
			 	\end{bmatrix}
		 			=
				\begin{bmatrix}
					c_1\\ 
					c_2\\
					c_3\\
				\end{bmatrix}$\\
		\end{addmargin}
		\begin{addmargin}[2em]{0em}
				Write the solution y as a matrix $A = S^{-1}$ times the vector $c$. Are the columns of $S$ independent or dependent?\\
		\end{addmargin}
	
	\end{addmargin}

	\begin{addmargin}[5em]{5em}
		\textcolor{blue} {\quad In order to compute $A = S^{-1}$, we can instead compute $SS^{-1} = I$. Using Gaussian elimination, we could computed A by creating an augumented matrix that has S on its left part and indentity matrix on the right part. Turning the left part indentity matrix will leave the right part become $s^{-1}$
			\begin{figure}[h]
				\centering
				\textcolor{blue}{
					$ \begin{bmatrix}
					1 & 0 & 0 & \aug & 1 & 0 & 0\\
					1 & 1 & 0 & \aug & 0 & 1 & 0\\
					1 & 1 & 1 & \aug & 0 & 0 & 1\\
					\end{bmatrix}$}
			\end{figure}
		\begin{center}
			Substract the $3^{rd}$ row by the $2^{nd}$ and substract the $2^{nd}$ row by the $1^{st}$ row gives us
		\end{center}	
		\begin{figure}[h]
			\centering
			\textcolor{blue}{
				$ \begin{bmatrix}
					1 & 0 & 0 & \aug & 1 & 0 & 0\\
					0 & 1 & 0 & \aug & -1 & 1 & 0\\
					0 & 0 & 1 & \aug & 0 & -1 & 1\\
				\end{bmatrix}$}
		\end{figure}	
		\quad As we can see, the left part is already the identity, so
		\begin{center}
				$A = S^{-1} = \begin{bmatrix}
				1 & 0 & 0\\
				-1 & 1 & 0\\
				0 & -1 & 1\\
			\end{bmatrix}$
		\end{center}
			\quad Therefore, $y = AC = \begin{bmatrix}
			1 & 0 & 0\\
			-1 & 1 & 0\\
			0 & -1 & 1\\
			\end{bmatrix}
			\begin{bmatrix}
				c_1\\ 
				c_2\\
				c_3\\
			\end{bmatrix}$.}	
		\textcolor{blue}{
		\quad By the rule of matrix multiplication, we can get
				\begin{center}
				$\begin{cases}
					y_1 = c1\\
					y_2 = -c1 + c2\\
					y_3 = -c_2 + c3\\
				\end{cases}$
			\end{center}}
		\textcolor{blue}{The columns in S are independent because there is no way to construct one column using other two columns in S.\\}
	\end{addmargin}

%problem 1 ends here
\clearpage

%problem 2 starts from here
\qquad p.41 \#5 
 \begin{align*}
	x + y + z &= 2\\
	x + 2y + z &= 3\\
	2x + 3y + 2z & = 5
 \end{align*}
\textcolor{blue}{
\indent (Fill in the blanks question) If $x,y,z$ satisfy the first two equations then they also satisfy the $3_{rd}$ equation because the $3_{rd}$ equation could be obtained by adding the $1_{st}$ equation and the $2_{nd}$ equation.}


	\begin{figure}[h]
		\centering
		\textcolor{blue}{
		$\begin{bmatrix}
			x\\
			y\\
			z
		\end{bmatrix}
		=
		\begin{bmatrix}
			0\\
			1\\
			1
		\end{bmatrix}	
		=
		\begin{bmatrix}
			1\\
			1\\
			0
		\end{bmatrix}
		=
\		\begin{bmatrix}
			30\\
			1\\
			-29
		\end{bmatrix}$}
	\end{figure}

%problem 2 ends here

Axler
%problem 3 starts from here

	 \qquad p.11 \#3 Suppose a and b are real numbers, not both 0. Find real numbers c and d such that $\frac{1}{a+bi} =  c+di.$
	 \textcolor{blue}{
	 \begin{center}
	 	$\frac{1}{a+bi}\times\frac{a-bi}{a-bi} = \frac{a-bi}{a^2-bi^{2}} = \frac{a-bi}{a^2+b^2}$
	 \end{center}
 	 \begin{center}
 	 	$c = \frac{a}{a^2+b^2}, d = \frac{-b}{a^2+b^2}$
 	 \end{center}}
  
  	 \qquad p.11 \#3 Find two distinct square roots of i.
  	 \textcolor{blue}{
  	 	\begin{center}
  	 		assume $\sqrt{i} = a+bi$\\
  	 		squaring both sides gives us $i = a^2 + 2abi - b^2$\\
  	 		Since the imaginary parts are equal, $2ab = 1, a^2 = b^2$\\
  	 		So $a = b = \pm \frac{\sqrt{2}}{2}.$\\
  	 		Therefore the two square roots of i are $\frac{\sqrt{2}}{2} + \frac{\sqrt{2}}{2}i$ and $- \frac{\sqrt{2}}{2} - \frac{\sqrt{2}}{2}i$
  	 	\end{center}}
%problem 3 ends here
\end{document}