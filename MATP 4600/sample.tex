%LaTeX sample file for Foundations of Analysis
%Everything following a % sign on a line is a comment
%LaTeX commands start with a \
%The delimiters {,} are used to group text together
%The delimiters [,] are frequently used for optional arguments

%The next line specifies the style of document to produce and asks for 12pt type
\documentclass[12pt]{article}   


%This specifies a larger left margin to permit grader remarks to be added.
\usepackage[letterpaper,textwidth=5in,right=2.5in,textheight=9in]{geometry}

\usepackage{amsmath,amssymb}





%The \begin{document} starts the actual document itself. Everthing above this line is just to set styling.

\begin{document}
%So here is where we start the document


\noindent{\bf Your Name}   % \noindent means to not indent this line.

\noindent{\bf Homework 1, Foundations of Analysis, Fall 2016}  %\bf means what is in the { } is bold face


You just type whatever you want here. Generally, LaTeX doesn't care
where you break lines in a paragraph. It will align things correctly
automatically. It also could care less about how many spaces you leave
between words. It will treat repeated spaces as just one space.

A blank line will start start a new paragraph, and multiple blank
lines are treated as just one blank line.


LaTeX automatically determines spacing and styling. Just concentrate
on content and let LaTeX make it look right. 

Text inbetween dollar signs are in ``math mode'' where math commands that
make a variety of math symbols are recognized. Here is how you might type the
some problem statements.


\noindent {\bf Suggested Problem} Suppose $a,b,c,d$ are real numbers.
Prove that if  $a< b < c < d$ then $(a,c) \cap (b,d) = (b,c).$


\noindent {\bf Section 2.1, Problem 23b} We want to show 
$ U \setminus (A \setminus B) = ( U \setminus A ) \cup B$

\bigskip %some extra vertical space

Here are some more math symbols that can be used in math mode

\begin{enumerate}  %Start an enumerated list, with each entry specified by \item
\item Intersection: $\cap$ or $\bigcap$
\item Union: $\cup$ or $\bigcup$
\item Setminus: $\setminus$
\item Logical Negation: $\sim $
\item Empty set: $\emptyset$
\item x is an element of A:    $x \in A$
\item p implies q: $p \implies q$
\item Set builder notation: $(-\infty,2) = \{x : x \in \mathbb{R}  \text{ and } x < 2 \}$
\end{enumerate} %End the enumerated list

\vfill %Fills vertical space 

\begin{center} (continued) \end{center}


\newpage % This starts a new page. 

For math mode, if you use two dollar signs, instead of one, at the beginning and the
end of a math mode expression, the equation displays centered on a
separate line.  LaTeX automatically includes extra spacing between
lines to allow for ``taller expressions''.  Below is how you can
typeset problem 26a in section 2.1. 
$$ B \cup \left[ \bigcap_{j \in J}  a_j \right] = \bigcap_{j \in J} \left( B \cap A_j \right) $$
Note the syntax in front of the brackets [ and ]. Using the ``slash left'' 
or ``slash right'' in front of these brackets tells LaTeX that you want 
it to automatically figure out the correct
height of these based on whatever they are is enclosing.
The same thing can be done for braces and parenthesis.

\bigskip

Below is how you make bold face and italics. Note that you need
braces around the items whose font you want to change.

\bigskip

\noindent{\bf Most importantly,} if you want to know how to do {\it ``something''} in LaTeX, you can usually just search the web for: {\it ``LaTeX something''}


%End the document with \end{document}
\end{document}

