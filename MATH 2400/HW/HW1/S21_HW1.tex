% def.tex

% File containing LaTeX macros.
% Has grown by accretion; can be better organized.

\documentclass{article}
\usepackage{amsmath}
\usepackage{amssymb}
\usepackage{bm}
\usepackage{graphicx}
\usepackage{epstopdf}
\DeclareGraphicsRule{.tif}{png}{.png}{`convert #1 `basename #1 .tif`.png}
\usepackage{color}
\usepackage{pdfsync}
\pagestyle{plain}

%\pagestyle{empty}

\textheight 9 true in
\textwidth 6.5 true in
\hoffset -.75 true in
\voffset -.75 true in
 \mathsurround=2pt  \parskip=2pt
\def\crv{\cr\noalign{\vskip7pt}} 

\def\a{\alpha } \def\b{\beta } \def\d{\delta } \def\D{\Delta } \def\e{\epsilon }
\def\g{\gamma } \def\G{\Gamma} \def\k{\kappa} \def\l{\lambda } \def\L{\Lambda }
\def\th{\theta } \def\Th{\Theta} \def\r{\rho} \def\o{\omega} \def\O{\Omega}
\def\ve{\varepsilon} 

\def\sA{{\cal A}} \def\sB{{\cal B}} \def\sC{{\cal C}} \def\sE{{\cal E}} \def\sI{{\cal I}}
\def\sR{{\cal R}} \def\sF{{\cal F}} \def\sG{{\cal G}} \def\sM{{\cal M}}
\def\sT{{\cal T}} \def\sH{{\cal H}} \def\sD{{\cal D}} \def\sW{{\cal W}}
\def\sL{{\cal L}} \def\sP{{\cal P}} \def\s{\sigma } \def\S{\Sigma}
\def\sU{{\cal U}} \def\sY{{\cal Y}}

\def\gm{\gamma -1}
\def\summ{\sum_{j=1}^4}

\def\bb{{\bm b}} \def\yb{{\bm y}}
\def\ub{{\bm u}}  \def\xb{{\bm x}} \def\vb{{\bm v}} \def\wb{{\bm w}}
\def\omegab{{\bm \omega}} \def\rb{{\bm r}} \def\ib{{\bm i}} \def\jb{{\bm j}}
\def\lb{{\bm l}} \def\kb{{\bm k}} \def\Ab{{\bm A}} \def\fb{{\bm f}} \def\Ub{{\bm U}}
\def\Fb{{\bm F}} \def\nb{{\bm n}} \def\Db{{\bm D}} \def\eb{{\bm e}}
\def\gb{{\bm g}}  \def\hb{{\bm h}} \def\Yb{{\bm Y}} \def\Rb{{\bm R}} 

\def\As1{{\bf {\cal A}}_1}\def\DO{{\cal D}_0} \def\UO{{\cal U}_0}
\def\ie{{\it{i.e.}}}

\def\ubbar{{\bf {\bar{u}}}} \def\sbar{{\bar{\sigma }}} \def\ubar{{\bar{u}}}  
\def\abar{{\bar{a}}} \def\vbar{{\bar{v}}}  \def\rbar{{\bar{\rho}}}
\def\pbar{{\bar{p}}} \def\ebar{{\bar{e}}} \def\Tbar{{\bar{T}}}
\def\bbar{{\bar{\beta}}} \def\Mbar{{\bar{M}}}  \def \sMbar{{\bar{\cal M}}}
\def\Ebar{{\bar{E}}} \def\sMbar{{\bar{\cal M}}}
\def\sPbar{{\bar{\cal P}}} \def\xbar{{\bar{x}}}

\newcommand{\pdv}[2]{\frac{\partial#1}{\partial#2}}
\newcommand{\dv}[2]{\frac{d#1}{d#2}}
\newcommand{\ord}[2]{#1^{(#2)}}
\newcommand{\vct}[1]{\vec{#1}}

 \newcommand{\bc}{\begin{center}}
 \newcommand{\ec}{\end{center}}
 
 \newcommand{\bq}{\begin{equation}}
 \newcommand{\eq}{\end{equation}}
 
 \newcommand{\beqs}{\begin{eqnarray}}
 \newcommand{\eeqs}{\end{eqnarray}}
 
 \newcommand{\beqa}{\begin{eqnarray*}}
 \newcommand{\eeqa}{\end{eqnarray*}}
 
 \newcommand{\ol}{\overline}
 \newcommand{\ul}{\underline}
 
 \newcommand{\dint}{{\int \!\! \int \!\!}}
 \newcommand{\tint}{{\int \!\! \int \!\! \int \!\!}}
 
 \newcommand{\bfig}{\begin{figure}}
 \newcommand{\efig}{\end{figure}}
 
 \newcommand{\cen}{\centering}
 \newcommand{\n}{\noindent}
 
 \newcommand{\btab}{\begin{table}}
 \newcommand{\etab}{\end{table}}
 
 \newcommand{\btbl}{\begin{tabular}}
 \newcommand{\etbl}{\end{tabular}}
 
 \newcommand{\bdes}{\begin{description}}
 \newcommand{\edes}{\end{description}}
 
 \newcommand{\benum}{\begin{enumerate}}
 \newcommand{\eenum}{\end{enumerate}}
 
 \newcommand{\bite}{\begin{itemize}}
 \newcommand{\eite}{\end{itemize}}
 
 \newcommand{\cle}{\clearpage}
 \newcommand{\npg}{\newpage}
 
 \newcommand{\bss}{\begin{singlespace}}
 \newcommand{\ess}{\end{singlespace}}
 
 \newcommand{\bhalf}{\begin{onehalfspace}}
 \newcommand{\ehalf}{\end{onehalfspace}}
 
 \newcommand{\bds}{\begin{doublespace}}
 \newcommand{\eds}{\end{doublespace}}
 
 \newcommand{\eps}{\mbox{$\epsilon$}} 
 \newcommand{\stilde}{\mbox{$\tilde s$}} 
 \newcommand{\shat}{\mbox{$\hat s$}} 

 \newcommand{\blue}{\color{blue}}
 \newcommand{\red}{\color{red}}
 \newcommand{\magenta}{\color{magenta}}
 \newcommand{\green}{\color{green}}
 \newcommand{\nc}{\normalcolor}



\def\arg{{\rm{arg}}}
\def\Arg{{\rm{Arg}}}
\def\imath{i}
\def\Log{{\rm{Log}}}
\def\erfc{{\rm{erfc}}}
\def\Oh{\mathcal{O}}
\def\oh{\mathsf{o}}

\def\Res{{\rm{Res}}}

\def\fl{{\rm{fl}}}

\def\Xint#1{\mathchoice
{\XXint\displaystyle\textstyle{#1}}%
{\XXint\textstyle\scriptstyle{#1}}%
{\XXint\scriptstyle\scriptscriptstyle{#1}}%
{\XXint\scriptscriptstyle\scriptscriptstyle{#1}}%
\!\int}
\def\XXint#1#2#3{{\setbox0=\hbox{$#1{#2#3}{\int}$ }
\vcenter{\hbox{$#2#3$ }}\kern-.6\wd0}}
\def\ddashint{\Xint=}
\def\dashint{\Xint-}

%\numberwithin{equation}{section}
\pagestyle{empty}
\begin{document}

\begin{center}
\large{ MATH-2400 \hspace{.25in}  INTRODUCTION TO DIFFERENTIAL EQUATIONS \hspace{.25in}SPRING 2021\\ Homework-1 (subjective) \\ Assigned Thursday January 28, 2021 \\ Due 12 Noon, Friday February 5, 2021}\end{center}
\textcolor{blue}{Wang Xinshi 661975305}\\
\bigskip
\n\ul{NOTES}
\benum
\item 
Practice problems listed below and taken from the textbook are for your own practice, and are not to be turned in.
\item 
Legible, handwritten solutions will be acceptable, but the use of a typesetting system such as LaTeX is strongly recommended.  \blue{\bf Do not turn in your rough attempt at solving a problem; once you have worked out the solution, copy it neatly or typeset it before submission, after removing all false starts.}\nc
\item 
Please write your solutions clearly and coherently, with the work displayed in a sequential manner and sufficient explanation provided so that your strategy and approach are transparent to the reader. 
\item 
Figures, if any, should be neatly drawn by hand, properly labelled and captioned.  
\item 
The assignment is to be submitted electronically to LMS  as a single pdf file.  Be sure that the pages are properly oriented and well lighted.  \blue{\bf Please do not e-mail your homework submission to the TAs or the instructors.}\nc
\eenum

\bigskip


\bc {\bf Practice Problems from the textbook (not to be turned in)} \ec

\n Exercises from Chapter 1, pages 4-6: 1(b,c,g), 2(g,i), 3(f), 5(d), 6.

\n Exercises from Chapter 2, pages 12-14: 1(c,f,l,m), 2(d,f,h), 3(f,g,h), 4(c,d).


\bc {\bf Subjective part: problems to be turned in} \ec

\begin{enumerate}

% Problem 1
\item (20 points)  Solve the IVP
\begin{equation*}
y'=2y^2(1-t), \quad y(0)=1.
\end{equation*}
What is the $t$-interval in which the solution is valid?
\nc
\\

\text{\bf Solution:}
\begin{align*}
	y' &= 2y^2(1-t)\\
	\dfrac{1}{2y^2} dy &= (1-t) dt\\
	\int \dfrac{1}{2y^2} dy &= \int (1-t) dt\\
	-\dfrac{1}{2y} & = t - \dfrac{1}{2}t^2+C\\
	2y &= - \dfrac{1}{t-\dfrac{1}{2}t^2}+C\\
	y &= - \dfrac{1}{2t-t^2+C}\\
\end{align*}

Substitute the IC y(0) = 1 in gives us $\dfrac{1}{C} = 1$. Therefore $C = 1$. Thus $y =  \dfrac{1}{(t-1)^2}$. In order for the solution to be valid, the demoninator must not be 0. Thus $(t-1)^2 \neq 0$ and therefore the solution is valid in interval $(-\infty,1)\cup(1,\infty)$.

\newpage
% Problem 2
\item (20 points) Consider the DE 
\begin{equation*}
2y' + y^3 \sin t = 0. 
\end{equation*}
\benum
\item Find all solutions.
\begin{align*}
	2y'+y^3sint &= 0\\
	2y' &= -y^3sint\\
	-\dfrac{2y'}{y^3} &= sint\\
	\int -\dfrac{2}{y^3}dy &= \int sint dt\\
	\dfrac{1}{y^2} &= -cost+C\\
	y^2 &= \dfrac{1}{C-cost}\\
	y &= \dfrac{1}{\sqrt{C-cost}}
\end{align*}
\item Find in explicit form the solution for the initial condition $y(0) = 1/2.$
\text{\bf Solution:} Substitute 0 in gives us $\dfrac{1}{\sqrt{C}} = \dfrac{1}{2}$. Thus $C = 4$ and $y = \dfrac{1}{\sqrt{4-cost}}$.
\item Find in explicit form the solution for the initial condition $y(0) = -1/2.$
\text{\bf Solution:} Substitute 0 in gives us $\dfrac{1}{\sqrt{C}} = -\dfrac{1}{2}$. There are no solution in real numbers.
\eenum

\newpage

% Problem 3
\item (20 points) Find, in implicit form, the solution of the IVP 
\begin{equation*}
\frac{dv}{du} + \frac{e^{u+v} + e^{-u+v}}{e^v+e^{-v}} = 0, \quad v(0) = 0.
\end{equation*}
\nc

\text{\bf Solution:}\\
\begin{align*}
	\dfrac{dv}{du} &= -\dfrac{e^{u+v}+e^{-u+v}}{e^v+e^{-v}}\\
	\dfrac{dv}{du} &= -\dfrac{e^v(e^{v}+e^{-u})}{e^v(1+e^{-2v})}\\
	\dfrac{dv}{du} &= -\dfrac{e^{v}+e^{-u}}{1+e^{-2v}}\\
	1+e^{-2v} dv &= 1+e^{-2v} du\\
	\int 1+e^{-2v} dv &= \int 1+e^{-2v} du\\
	v - \dfrac{1}{2}e^{-2v} &= -e^{u}-e^{-u}+C 
\end{align*}
\end{enumerate}





\end{document}